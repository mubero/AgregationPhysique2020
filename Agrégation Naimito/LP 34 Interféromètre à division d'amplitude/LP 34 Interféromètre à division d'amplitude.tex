\documentclass[12pt,prb,aps,epsf]{report}
\usepackage[utf8]{inputenc}
\usepackage{amsmath}
\usepackage{amsfonts}
\usepackage{amssymb}
\usepackage{graphicx} 
\usepackage{latexsym} 
\usepackage[toc,page]{appendix}
\usepackage{listings}
\usepackage{xcolor}
\usepackage{soul}
\usepackage[T1]{fontenc}
\usepackage{amsthm}
\usepackage{mathtools}
\usepackage{setspace}
\usepackage{array,multirow,makecell}
\usepackage{geometry}
\usepackage{textcomp}
\usepackage{float}
%\usepackage{siunitx}
\usepackage{cancel}
%\usepackage{tikz}
%\usetikzlibrary{calc, shapes, backgrounds, arrows, decorations.pathmorphing, positioning, fit, petri, tikzmark}
\usepackage{here}
\usepackage{titlesec}
%\usepackage{bm}
\usepackage{bbold}

\geometry{hmargin=2cm,vmargin=2cm}

\begin{document}
	
	\title{LP 34 Interféromètre à division d'amplitude}
	\author{Clément}
	
	\maketitle
	
	\tableofcontents
	
	\pagebreak
	
	
\subsubsection{prérequis}
interf a divison du front d'onde\\
Optique géométrique\\
interférences

\section{Introduction}
\subsection{Cohérence spatiale de la source}
Schéma type fentes d'Young : source primaire $S_0$ qui génère deux sources secondaires cohérentes $A_1$ et $A_2$.\\
Calcul de 
\begin{eqnarray}
\delta (S,M) = (SA_1M)-(SA_2M) = \delta (S_0,M) -S_0A1 + S_0A2 + SA_1 - SA_2\\
\Delta \delta = [S_0A2-SA_2] - [S_0A_1-SA_0]\\
S_0A_2 = S_0S + SA_2\\
... \\
\mathrm{Finallement}\; \Delta \delta = \vec{S_0S}.(\vec{u}_1-\vec{u}_2)
\end{eqnarray}
L'intérêt d'un interféromètre à division d'amplitude va donc être que $\vec{u}_1-\vec{u}_2=\vec{0}$ ce qui va ainsi permettre que l'interférogramme ne souffre pas de l'extension de la source.

\section{Interféromètre de Michelson}
\subsection{Principe}
Définition de ce qu'est une lame semi réfléchissante, permettant ici de séparer le faisceau incident en deux faisceaux distinct qui vont suivre des chemins différents avant d'interférer.\\
Schéma de principe.\\
La lame séparatrice n'étant pas infiniment mince elle va absorber une partie du rayonnement, et va de plus engendrer des problèmes du fait que cela va dépendre de la longueur d'onde.\\

\subsection{Source ponctuelle}
Dans le cas d'une source ponctuelle : tracé de rayons sur le schéma.\\
Analogie avec une lame d'air en considérant le symétrique par rapport à la lame séparatrice du second miroir.\\
\subsection{Lame d'air}
Cas où les lames sont parallèles : Schéma avec tracé de rayon, dans l'air, et calcul de la différence de marche entre rayon réfléchi et rayon réfracté 2 fois et réfléchi 1 fois.
\begin{eqnarray}
\delta = 2e\cos(r)
\end{eqnarray}
Illustration en direct avec les michelson réglé en lame d'air proche du contact optique avec la lampe à sodium.\\
Cas du doublet du sodium : tracé de la DSP en fonction de $\lambda$ ou $\nu$, puis tracé des éclairements dus à chacune des longueur d'onde, explication de la perte de contraste au point où $\delta$ est tel que les deux éclairements sont en opposition de phase. $\rightarrow$ application possible du michelson réglé en lame d'air pour mesurer la différence de longueur d'onde entre les deux raiens du doublet.

\subsection{Lame d'air}
Schéma + localisation des interférence. Calcul de la différence de marche  $\delta$ en fonction de l'angle entre les deux lames $\alpha$, $\delta \simeq 2e(x) = 2 \alpha x$ avec x la distance entre le point d'intersection des deux lames et le point d'incidence du rayon.\\
Illustration avec le Michelson réglé en coin d'air.

\section{Fabry-Pérot}

\section*{Conclusion}
Intéret de ce type d'interféromètre par rapport à l'extension spatiale des sources.\\
Intéret pour mesurer précisément des longueurs.\\
Ouverture sur la détection des ondes gravitationnelles avec Virgo qui un combiné Michelson Frabry-Pérot

\section{Questions}
Qu'est ce que la localisation des interférences exactement ? Pourquoi sont elles localisée, dans le cas du coin d'air, derrière les séparatrices ?\\

\section{Remarques}
Utiliser un vocabulaire adapté : pour montrer que la mesure du doublet du sodium est précise : définir le pouvoir de résolution $\frac{\lambda}{\Delta \lambda}$.\\
En 40 minutes : a t'on le temps de faire un schéma pour une simple différence de chemin optique ??? $\rightarrow$ utiliser un schéma projeté.\\
Attention à l'accord entre les figures projetées et les schémas construits au tableau.\\
Rédiger une phrase de construction pour chaque paragraphe et la réécrire encore et encore jusqu'à ce qu'elle soit parfaite.\\
Enlever les "on pourrait", soit on le fait, soit on ne l'évoque pas.\\
Tant qu'à manipuler le Michelson : il faut que la figure soit optimale.\\
Ne pas parler de la séparatrice ou au moins ne pas s'y attarder, puis aller directement à l'analogie Miquelson $\leftrightarrow$ Lames d'air.\\
Parler du Fabry Pérot en ouverture et énoncer une application moderne possible.




\end{document}