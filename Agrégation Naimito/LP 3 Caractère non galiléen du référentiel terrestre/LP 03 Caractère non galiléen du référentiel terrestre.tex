\documentclass[12pt,prb,aps,epsf]{report}
\usepackage[utf8]{inputenc}
\usepackage{amsmath}
\usepackage{amsfonts}
\usepackage{amssymb}
\usepackage{graphicx} 
\usepackage{latexsym} 
\usepackage[toc,page]{appendix}
\usepackage{listings}
\usepackage{xcolor}
\usepackage{soul}
\usepackage[T1]{fontenc}
\usepackage{amsthm}
\usepackage{mathtools}
\usepackage{setspace}
\usepackage{array,multirow,makecell}
\usepackage{geometry}
\usepackage{textcomp}
\usepackage{float}
%\usepackage{siunitx}
\usepackage{cancel}
%\usepackage{tikz}
%\usetikzlibrary{calc, shapes, backgrounds, arrows, decorations.pathmorphing, positioning, fit, petri, tikzmark}
\usepackage{here}
\usepackage{titlesec}
%\usepackage{bm}
\usepackage{bbold}

\geometry{hmargin=2cm,vmargin=2cm}

\begin{document}
	
	\title{LP 03 Caractère non galiléen du référentiel terrestre}
	\author{Maxime}
	
	\maketitle
	
	\tableofcontents
	
	\pagebreak
	
	
\subsection*{Pré-requis}
- Mécanique du point \\
- Oscillateurs harmoniques
	
\section{Introduction}
Définition référentiel. \\
Notion de référentiel galiléen, mise en valeur du fait que les référentiels réels ne le sont pas en général.
\section{Référentiels galiléens et non galiléens}
\subsection{Définitions}
\paragraph{Référentiel galiléen}
Référentiel dans lequel on vérifie le principe d'inertie.
\paragraph{Référentiel non galiléen}
Référentiel dans lequel le principe d'inertie n'est pas valable( ref accéléré ou en rotation par rapport à un ref galiléen)

\subsection{Référentiels usuels}
Référentiel de Copernic $R_0$, ayant pour centre le soleil et dont les axes sont définis à partir de 3 étoiles lointaines, galiléen pour des durées usuelles.\\
Référentiel géocentrique $R_G$, centré sur la terre et avec les mêmes axes que $R_0$.\\
Référentiel terrestre $R_T$, placé à la surface de la terre, avec l'axe Oz représentant l'altitude et l'axe oy pointant vers le nord.

\subsection{Lois de la dynamique}

Transformation des vitesses
\begin{eqnarray}
\frac{d\vec{A}}{dt}|_R = \frac{d\vec{A}}{dt}|_{R'}  + \vec{\Omega}_{R'/R}\times \vec{A}\\
\Rightarrow \vec{V}(M)|_R = \vec{V}(O')|_R + \vec{V}(M)|_{R'}+ \vec{\Omega}_{R'/R}\times \vec{O'M}
\end{eqnarray}
On en déduit la composition des accélérations 
\begin{eqnarray}
\vec{a}|_R = \vec{a}|_{R'} + \vec{a}_e + \vec{a}_c\\
\vec{a}_e = \vec{a}_{R'/R} + \frac{d\vec{\Omega}_{R'/R}}{dt}\times \vec{O'M} + \vec{\Omega}_{R'/R} \times (\vec{\Omega}_{R'/R} \times \vec{O'M})
\end{eqnarray}
$\rightarrow$ notion de forces de coriolis et d'inertie d'entrainement.

\subsection{Exemples simples}
\paragraph{Translation rectiligne uniforme}
\begin{eqnarray}
\vec{V}_{R'/R} = \vec{cste} \; \& \; \vec{\Omega}_{R'/R} = \vec{0}\; \Rightarrow \; \vec{a}_e = \vec{a}_c = \vec{0}
\end{eqnarray}
\subsubsection*{Référentiel uniformément accéléré}
$\rightarrow$ on observe ici une force d'inertie d'entrainement.\\
Exemple de la voiture, où l'on est collé au siège lors du démarrage.

\subsubsection*{Référentiel en rotation uniforme}
Notion de force centrifuge dans le cas d'un point fixe dans le référentiel en rotation, qui est en fait la force d'inertie d'entrainement dans ce cas particulier ($\vec{a}_{R'/R}=\vec{0}$ et $ \frac{d\vec{\Omega}_{R'/R}}{dt}\times \vec{O'M}=\vec{0}$)

\section{Observation du caractère non galiléen du référentiel terrestre}
\subsection{Marée océaniques}
\subsubsection*{Définition du poids}

PFS :
\begin{eqnarray}
m\vec{G} - m\vec{a}_e - m\vec{a}_c - \vec{P} = \vec{0}
\end{eqnarray}
Or dans le cas de la Terre $\vec{a}_c=\vec{0}$, donc 
\begin{eqnarray} 
\Rightarrow \vec{P} = m\vec{G} - m\vec{a}_e = m\vec{g}\; \Rightarrow \; \vec{g} = \vec{G} - \vec{a}_e
\end{eqnarray}
Si on cherche à exprimer l'accélération d'inertie d'entrainement su Terre on obtient :
\begin{eqnarray}
\vec{a}_e &=& \vec{a}(T)|_{R_0} + \vec{\Omega}_T\times( \frac{d \vec{TM}}{dt}|_{R_T} + \vec{\Omega}_T\times \vec{TM})\\
&=& \vec{a}(T)|_{R_0} + \vec{\Omega}_T\times(\vec{\Omega}_T\times \vec{TM})
\end{eqnarray}
On exprime toutes les composantes de l'accélération de la pesanteur, dues la terre, la lune et le soleil et à la rotation de la terre autour de son axe.\\
Schéma interaction gravitationnelle terre-lune. On y note M un point de la surface et T le centre de la Terre. \\
On obtient 
\begin{eqnarray}
|| \vec{G}_a(M)-\vec{G}_a(T)|| \simeq \frac{2GM_LR_T}{D_{TL}^3}
\end{eqnarray}
Remarque concernant la période des marrées, plus impact du soleil menant aux "vives eaux" et "eaux calmes" selon si le soleil compense ou s'ajoute à l'effet de la lune.\\
V Pérez p106

\subsection{Pendule de Foucault}
Pérez p109\\
On applique le PFD au pendule que l'on considère comme parfait (liaison pivot idéale) en considérant le référentiel terrestre comme non galiléen et on obtient l'équation du pendule simple à un terme d'accélération de Coriolis près.\\
Données et application chiffrée au pendule de Foucault situé à Paris.

\subsection{Déviation vers l'Est}
Déviation vers l'Est d'un objet lors de sa chute libre sur Terre due à la force de Coriolis.\\
Pérez p113

\section*{Questions}
Comment établir un "vrai" référentiel galiléen ? Quel serait le référentiel inertiel idéal ?\\

Concernant le pendule de Foucault, pourquoi prend on un pendule de très grande longueur ?\\
$\rightarrow$ pour pouvoir observer un décalage conséquent du pendule en un aller retour (plus facile de percevoir l'angle)\\

Comment établit on la période de rotation du plan du pendule de Foucault ?\\

Quelle est l'amplitude des marrées ?\\

Pourquoi néglige t'on accélération d'entrainement dans le problème des marrées ?\\

Y a t'il d'autres applications du caractère non galiléen du référentiel terrestre ?\\

Pourquoi le référentiel de Copernic n'est il pas rigoureusement galiléen ?

\section*{Remarques}
Ne pas mettre les marées dans cette leçon : trop risqué, et puis pas tellement lié au caractère non galiléen du référentiel non galiléen de la Terre, garder les deux autres exemples seulement. Les garder pour les questions.\\
Pour le cas du cyclone on écrit Navier-Stokes avec les termes d'accélération de coriolis et d'entrainement.
\paragraph{Or} \begin{eqnarray}
\vec{a}_e = \vec{\Omega} \times \vec{\Omega} \times \vec{r} = -\vec{\nabla}(\frac{1}{2}\Omega^2s^2)
\end{eqnarray}
où $s$ est la distance à l'axe de rotation. On peut donc ajouter l'accélération d'entrainement dans le terme $\vec{grad}P$ pour obtenir une pression effective $\Pi$. En introduisant les longueurs caractéristiques d'un cyclone on voit que le nombre de Rosby est petit, on en déduit que l'équation se résume à
\begin{eqnarray}
\vec{e}_z \times \vec{u} = -\vec{\nabla}\Pi
\end{eqnarray} 
qui peut se résoudre comme
\begin{eqnarray}
\vec{u} = \vec{e}_z \times \vec{\nabla}\Pi
\end{eqnarray} 
Donc l'air va tourner et forme un cyclone. Un tel écoulement est dit géostrophique.\\
$\rightarrow$ exemple possible et assez simple à traiter pour cette leçon.\\

Le référentiel inertiel idéal est le référentiel comobile.\\ remarque : regarder ICRF : ref def à partir du barycentre du système solaire et dont les axes pointent vers des quasars.\\
On peut connaître notre vitesse par rapport au référentiel comobile grâce au décalage vers le rouge (effet Doppler) du fond diffus cosmologique.\\

La force de coriolis assure la conservation du moment cinétique.\\

 autres exemples classiques :\\ x lancement des fusées à l'équateur.\\ x Usure asymétrique des rails de chemin de fer : le rail Est est plus usé que le rail Ouest pour les trains qui vont vers le Nord.\\ x (Lit des rivières asymétriques : à vérifier) 
 
\end{document}