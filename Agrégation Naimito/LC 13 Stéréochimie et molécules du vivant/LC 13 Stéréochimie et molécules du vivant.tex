\documentclass[12pt,prb,aps,epsf]{article}
\usepackage[utf8]{inputenc}
\usepackage{amsmath}
\usepackage{amsfonts}
\usepackage{amssymb}
\usepackage{graphicx} 
\usepackage{latexsym} 
\usepackage[toc,page]{appendix}
%\usepackage{listings}
\usepackage{xcolor}
\usepackage{soul}
\usepackage[T1]{fontenc}
\usepackage{amsthm}
\usepackage{mathtools}
\usepackage{setspace}
\usepackage{array,multirow,makecell}
\usepackage{geometry}
\usepackage{textcomp}
\usepackage{float}
\usepackage{cancel}
\usepackage{here}
\usepackage{titlesec}
\usepackage{bbold}

\geometry{hmargin=2cm,vmargin=2cm}

\begin{document}
	
	\title{LC 13 Stéréochimie et molécules du vivant}
	\author{Hugo}
	\date{Agrégation 2019}
	\maketitle
	
	\tableofcontents
	
	\pagebreak
	
\section{Représentation spatiale des molécules - Cram}
	Exemples du méthane et de l'éthane.\\
	Définition d'isomère.\\
	Stéréoisomérie de conformation\\
	
\section{Stereoisomérie de conformation}
\subsection{Représentation de Newman}
Représentation des deux exemples précédents.

\subsection{Etude énergétique - stabilité}
Cas éclipsé $E_p = -7\,kJ.mol^{-1}$, petit angle $E_p \simeq -10,2\,kJ.mol^{-1}$ et décalé : $E_p = -19.8\, kJ.mol^{-1}$.\\
Tracé de $E_p(\theta)$ $\rightarrow$ conformation décalée la plus stable.\\
Cas du butane.\\
Importance de ces aspects en biologie, notamment dans le cas des acides aminés.

\subsection{Les acides aminés - protéines}
Définition à partir d'un schéma.\\
Exemple de la glycine et de l'alanine.\\
Définition de liaison peptidique: liaison amide dans son ensemble
\begin{eqnarray}
&O&\\
&||&\\
&C& -N
\end{eqnarray}
Exemple à partir de protéines : cas du prion.\\
Cas de l'ADN.

\section{Stéréoisomères de configuration}
Définition.\\
Diagramme (ou arbre) avec les différents types d'isomères.

\subsection{Chiralité}
Notion de chiralité.\\
Exemple de la montre avec sa vis de réglage.\\
Notion de carbone asymétrique.

\subsection{Énantiomères}
\subsubsection{Cas d'un carbone asymétrique}
Définition d'énantiomère (S) ou (R). (initiales viennent du lation S pour gauche et R pour droite)\\
En général les récepteurs biologiques sont chiraux donc deux énantiomères n'auront pas les mêmes propriétés biologiques. (ex : pas la même odeur).

\subsubsection{Cas de deux carbones asymétriques}
Exemple de l'acide tartrique, configurations (2S,3S) et (2R,3R) qui ont des températures de fusion différentes.\\
Deux énantiomères peuvent avoir les mêmes proprités physiques, mais pas optiques.

\subsubsection{Activité optique}
Définition de levrogyre et dextrogyre, et de mélange racémique.

\subsection{Diastéréoisomères}
Cas de l'acide tartrique et de la treose.\\

\textbf{Manip : } Comportements différents de 2 diastéréoisomères : Acide maléique (Z) et acide fumarique (E), voir le document joint au dossier.\\

Comparaison des propriétés acides : voir correction de TP jointe.

\section*{Questions}
Comment fait on une solution tampon ?\\

Différence entre racémisation et épimérisation?\\
La racémisation va vers la racémisation du mélange.\\

Existe t-il d'autres types de chiralité dans le monde du vivant ? l'ADN est elle chirale ?\\
Oui : chiralité hélicoïdale, définie par (P) on visse le tire bouchon en allant à droite, (M) pour l'autre sens.\\

Qu'est ce que la chiralité planaire ?\\

Atropoisomérie ?\\
V Noyorori, Nols : Ldopa....\\

Pouvez vous représenter un l-acide aminé en représentation de Fisher ?\\
\begin{eqnarray}
&COOH&\hspace{1cm}\;(D)\\
&|&\\
H-&-C-&-NH_2\\
&|&\\
&R&
\end{eqnarray}

Réprésenter un D-oz ?
\begin{eqnarray}
&CHO&\\
&|&\\
H&-C-&OH\\
&|&\\
HO&-C-&H\\
&|&\\
H&-C-&OH \;\leftarrow\\
&|&\\
&CH_2OH&
\end{eqnarray}

Qu'est ce que faire une détermination de configuration absolue ?\\
Déterminer la structure 3D\\

Différence entre une synthèse stréréosélective et stéréospécifique ?\\

Qu'est ce qu'un acide aminé essentiel\\
AA que l'on ne peut synthétiser (l'humain) et que l'on doit aller chercher dans l'alimentation.

\section*{Remarques}
Enlever les points de fusion.\\
Plan bien lié.\\
Bon point : réitérer les mêmes. exemples au long du cheminement afin de voir différents aspects avec de mêmes molécules\\
Exemple pour les différences de chiralité : citer la poignée de main plutôt que la montre.\\
Possibilité d'évoquer la kétamine.\\
Regarder livre "chiralité" de Jeanne Crassous.
\end{document}