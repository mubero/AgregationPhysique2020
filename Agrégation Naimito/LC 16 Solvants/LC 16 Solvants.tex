\documentclass[12pt,prb,aps,epsf]{article}
\usepackage[utf8]{inputenc}
\usepackage{amsmath}
\usepackage{amsfonts}
\usepackage{amssymb}
\usepackage{graphicx} 
\usepackage{latexsym} 
\usepackage[toc,page]{appendix}
\usepackage{listings}
\usepackage{xcolor}
\usepackage{soul}
\usepackage[T1]{fontenc}
\usepackage{amsthm}
\usepackage{mathtools}
\usepackage{setspace}
\usepackage{array,multirow,makecell}
\usepackage{geometry}
\usepackage{textcomp}
\usepackage{float}
%\usepackage{siunitx}
\usepackage{cancel}
%\usepackage{tikz}
%\usetikzlibrary{calc, shapes, backgrounds, arrows, decorations.pathmorphing, positioning, fit, petri, tikzmark}
\usepackage{here}
\usepackage{titlesec}
%\usepackage{bm}
\usepackage{bbold}

\geometry{hmargin=2cm,vmargin=2cm}

\begin{document}
	
	\title{LC 16 Solvants}
	\author{Matthieu}
	
	\maketitle
	
	\tableofcontents
	
	\pagebreak
	
	
\subsection{Pré-requis}
Première année de prépa.\\
Classification, VSEPR, dosage, moment dipolaire, modèle de Lewis

\subsection{Introduction}
Omniprésence. Gros impact du solvant sur la réaction $\rightarrow$ choix important. Liaisons faibles (Van der Waals) entre le solvant et les molécules solvatées.

\section{Interactions faibles}
\subsection{Nécessité expérimentales}
Liaisons covalentes de l'ordre du kJ.mol$^{-1}$. Chaleur latente de vaporisation 100 x + faibles... il y a donc d'autres types de liaisons.

\subsection{Interactions de Van der Waals}
Certaines molécules possèdent un moment dipolaire $\Rightarrow$ interactions dipolaires. Il existe 3 types d'interactions de VdW : Keesom(moments dipolaires permanents), Debye(une des 2 molécules induit un moment dipolaire sur l'autre) et London(dipôles instantanés), qui sont fonction de la polarisabilité des molécules liées.\\
On observe bien cette fois que ces interactions ont des énergies du même ordre de grandeur que les chaleurs latentes associées.\\

Nécessité d'introduire un nouveau type de liaisons pour expliquer certaines températures de fusion anormalement élevées (au regard des liaisons de VdW)

\subsection{Liaison hydrogène}
Schéma type et principe. Énergie de 10 à 40 kJ.mol$^{-1}$.\\
Exemple des acides fumarique et maléïque : dans un cas seulement on peut former une liaison hydrogène intermoléculaire (dans l'autre elle est intramoléculaire).\\
Exemple du Kevlar.

Nous allons maintenant voir comment ces liaisons se manifestent à l'échelle macroscopique pour les solvants.

\section{Solvants}
\subsection{Définition}
Définir explicitement le terme solvant.

\subsection{Propriétés et classification}
énergie potentielle électrostatique
\begin{eqnarray}
U(r) \propto \frac{1}{r\varepsilon_r}
\end{eqnarray}
$\rightarrow$ Importance de la constante diélectrique du solvant. $\varepsilon_r$ caractérise le pouvoir dispersant.

\paragraph{Polarité}
Impact. On peut classer les familles de solvant par polarité :\\
eau > ac carbo > alcools > cétones > éthers > alcanes

\paragraph{Proticité}
Définition.

\subsection{Dissolution}
Dissoudre un composé consiste à le disperser dans le solvant, dont les molécules vont entourer, "solvater" les molécules de composé.\\

Il y a plusieurs étapes :
\paragraph{Ionisation} : $HCL = H^+  Cl^-$
\paragraph{Dispersion} : $H^+Cl^- = H^+ + Cl^-$
\paragraph{Solvatation} : les ions sont entourés par le solvant $H^+_{(aq)} + Cl^-_{(aq)}$.\\

Catégories de solvants : polaires protiques, polaires et aprotiques, apolaires et aprotiques... intérêt de chaque classe, et exemples.\\
En générale on solubilise bien une molécule avec un solvant du même type.\\

\textbf{Manip} : on solubilise du sel dans trois solvants différents : de l'eau, du cyclohexane et de l'éthanol. Seule l'eau parvient à dissoudre le sel qui est un solide ionique : se dissout bien dans un solvant polaire protique. Le cyclo est apolaire aprotique : très mauvais solvant pour le sel. Cas de l'éthanol plus ambigu : moins polaire que l'eau...

\subsection{Miscibilité}
Deux solvants de mêmes types se mélangent, mais ce n'est plus le cas pour deux solvants de type différents.

\paragraph{Manip :} \textit{Hachette 2nde 2010 p 211}. On met de la bétadine (diiode) dans de l'eau (le tout dans une ampoule à décanter) : elle diffuse, on ajoute du cyclohexane : on voit les deux solvant distinctement séparés par une interface (avec le diiode dans l'eau). On secoue en dégazant. Le cyclohexane est un bien meilleur solvant que l'eau pour le diiode : on constate après avoir secoué que (presque) tout le diode est allé dans le cyclohexane (si on a pas trop mis de bétadine).\\

Si on rajoute de la bétadine et que l'on secoue à nouveau on constate qu'il ya alors de la bétanine solvatée dans les deux solvants : il y a un équilibre de partage.

\section{Extraction liquide-liquide}
Principe. Expression du coefficient de partage.
\paragraph{Manip :} \textit{Chime organique expérimentale} de \textbf{Chavanne} p 153. On extrait l'acide propanoïque avec de l'ether.


\section*{Questions}
En première année, est ce la première fois que l'on parle de ces interactions ? des solvants ?\\
On a parlé des solvants en première mais c'est la première fois que l'on parle de liaisons VdW.\\

En quoi les chaleurs latentes nous montrent la nécessité de parler de liaisons VdW ?\\
Parce que les liaisons covalentes correspondent à des énergies 100x + grande.\\

La notion de chaleur latente a t-elle déjà été abordée ?\\
Oui au lycée, peut être sans la nommer ainsi.\\

Y a t'il un autre critère qui motive la recherche de liaisons autres que covalentes ?\\
Les molécules ont les mêmes propriétés après vaporisation puis condensation : les molécules et donc les liaisons covalentes sont intactes.\\

Concernant l'énergie potentielle électrostatique ne serait il pas plus facile pour les élèves de parler de force ? Quelle est la force associée ?
\begin{eqnarray}
F = \frac{Z_+Z_- e^2}{4\pi \varepsilon r^2 \varepsilon_r}
\end{eqnarray}

Pouvez vous justifier que le cyclohexane soit apolaire ?\\
Les liaisons qui composent la molécule ont des moments dipolaires nuls (ou quasi nuls).\\ 

Comment définit on le moment dipolaire d'une molécule ?\\
On fait la somme vectorielle des moments dipolaires des liaisons.\\

L'acétone est elle une molécule polaire ? Et quel est sont autre nom ?\\
propanone, et oui il suffit de la dessiner pour le voir.\\

L'éther diéthylique est il polaire ?\\
Oui si on applique VSEPR on voit que l'on a un oxygène central du type $AX_2E_2$ : la molécule est coudée et donc polaire.\\

Problèmes liés aux solvants ?\\
Toxicité et inflammabilité.\\

Quels sont les solvants à risque ?\\
Benzène, le cyclohexane, le toluene, le methanol.\\

Quels sont les conséquences sur les méthodes utilisées pour les manipuler ?\\
Il faut gérer leur élimination et prendre des précautions pour les manipuler.\\

Dans l'exemple choisi de la solvatation de HCl : quel est l'état de HCl ?\\
La molécule est solvatée, puis les ions se séparent, et ils sont ensuite dispersés puis solvatés, il faut donc écrire 
\begin{eqnarray}
HCl_{(aq)} \rightarrow (H^+Cl^-)_{(aq)} \rightarrow H^+_{(aq)} + Cl^-_{(aq)}
\end{eqnarray}

Pouvez vous montrer qu'un solvant polaire aprotique solvate moins un bien un anion qu'un cation ?\\

Qu'y at'il dans la bétadine ?\\ Du diiode et un polymère.\\

Était-ce important de mesurer les volumes pour la 2e manip ?\\
Non, du moins pas précisément, mais on veut tout de même que tout le diiode aille dans le cyclohexane la première fois.\\

Dans une ampoule à décanter, jusqu'où peut on remplir ?\\
La moitié.\\

Où faut il placer l'interface ?\\
Dans la partie large de l'ampoule pour pouvoir agiter au niveau de l'interface.

\section*{Remarques}
Il faut écourter la première partie, qui n'est pas vraiment pertinente : si c'est nouveau c'est trop rapide ... Si c'est des rappels c'est trop long et il y a trop de superflu comme les différents types d'interactions de VdW.\\

Plutôt parler de liaisons intermoléculaires que d'interaction faible car ce terme peut prêter à confusion.\\

Citer les dangers et inconvénients liés aux solvants.\\

Projeter les protocoles avec la flexcam ou un diapo.
\end{document}