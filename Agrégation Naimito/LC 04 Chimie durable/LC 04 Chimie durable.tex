\documentclass[12pt,prb,aps,epsf]{article}
\usepackage[utf8]{inputenc}
\usepackage{amsmath}
\usepackage{amsfonts}
\usepackage{amssymb}
\usepackage{graphicx} 
\usepackage{latexsym} 
\usepackage[toc,page]{appendix}
\usepackage{listings}
\usepackage{xcolor}
\usepackage{soul}
\usepackage[T1]{fontenc}
\usepackage{amsthm}
\usepackage{mathtools}
\usepackage{setspace}
\usepackage{array,multirow,makecell}
\usepackage{geometry}
\usepackage{textcomp}
\usepackage{float}
%\usepackage{siunitx}
\usepackage{cancel}
%\usepackage{tikz}
%\usetikzlibrary{calc, shapes, backgrounds, arrows, decorations.pathmorphing, positioning, fit, petri, tikzmark}
\usepackage{here}
\usepackage{titlesec}
%\usepackage{bm}
\usepackage{bbold}

\geometry{hmargin=2cm,vmargin=2cm}

\begin{document}
	
	\title{LC 04 Chimie durable}
	\author{Maxime}
	
	\maketitle
	
	\tableofcontents
	
	\pagebreak
	
	
\subsection{Pré-requis}
Réactions acido basiques et d'oxydo réduction\\
Synthèses\\
Chimie organique\\
Cinétique des réactions

\section{Les 12 principes de la chimie verte}
Établi récemment notamment dans un ouvrage américain.

\subsubsection{Éviter les déchets}
Définition de déchet : produit indésirable, dont on ne se sert pas.\\
Exemple du styrène où l'on créé HCl et $H_2$ qui sont ici les déchets.\\
Exemple de la synthèse du diméthylpropène, où l'on synthétise aussi de l'éthanol et du $NaBr$.\\
Il est souvent tout de même possible de valoriser les produits créés et non voulus au départ, on peut citer le cas du $CO_2$.

\subsubsection{Économies d'étapes et d'atomes}
On pourra pour ça optimiser le rendement.\\
Concept de l'utilisation atomique
\begin{eqnarray}
UA = \frac{M(produit\;desire)}{\sum \alpha_i M_i (reactifs)}
\end{eqnarray}
on donne un exemple (on peut reprendre celui du diméthylpropène, et regarder deux procédés de synthèse différents).

\subsubsection{Réduire la dangerosité}
On se doit de fabriquer des produits les plus saints possibles, tout en limitant la dangerosité des produits mis en jeux lors des synthèses.

\subsubsection{Réduire l'utilisation des substances auxiliaires}
Auxiliaires = Solvants(réaction, extraction, chromatographie) : espèces n'intervenant pas directement dans l'équation de synthèse.\\
Il existe des tables où sont référencés les différents solvants et donc les quels utiliser de préférence selon les cas.

\subsubsection{Efficacité énergétique}
Dans le cas des réactions cela se traduit par conditions de pression et température. Pour chauffer par exemple on peut utiliser différentes méthodes, comme les micro-ondes, ou la sonochimie qui utilise les ultrasons par exemple, il faut alors essayer de choisir les moins coûteuses.

\subsubsection{Ressources renouvelables}
Il faut essayer de se détacher des ressources fossiles, dont les stocks mettent des temps très longs à se renouveler. Il faut plutôt privilégier les ressources renouvelables. 

\subsubsection{Privilégier les catalyseurs}
Catalyseur : espèce n'intervenant pas explicitement à la réaction (n'apparaît pas dans l'équation de réaction) mais influence grandement la vitesse de celle ci.\\Exemple.

\subsubsection{Penser à la dégradation des produits}

\subsubsection{Surveiller les processus en temps réel}

\subsubsection{Prévention contre les risques}

\section{Manipulation illustrative : Synthèse de  l'éthanoate de linalyle}

Linalol + Acide éthanoïque  $\longrightarrow$ linalyle + eau\\
Cette réaction se fait en général avec un chauffage au bain marie, on se propose ici de le remplacer par un four à micro-ondes.

\section*{Questions}
Pouvez vous caractériser le produit obtenu lors de votre synthèse ?\\
Oui en utilisant l'indice de réfraction.\\

Où interviennent les énantiomères dans cette leçon ?\\
On va chercher à faire des synthèses énantiosélectives car la séparation des énantiomères est très coûteuse, et que l'énantiomère-déchets peut être compliqué à traiter.\\

Pouvez vous réécrire la réaction de synthèse du styrène ?\\

Quel est l'opération subie par le benzène ?\\
Une substitution.\\

Quel type de réaction doit on privilégier en chimie verte : addition, substitution ou élimination ?\\
Les additions et les substitutions, car elles produisent moins de déchets.\\

Comparer deux cas en regardant seulement l'utilisation atomique est elle pertinente ?\\
Non il y d'autres aspects tout aussi importants à regarder.\\

Que faut il regarder d'autre alors ?\\

Que faut il pour que la chimie verte s'impose ?\\
Il faut que les coûts deviennent plus intéressants que ceux de la chimie du pétrole.\\

Pourquoi la chimie du pétrole est elle si dure à détrôner ?\\
Car on la maîtrise à la perfection et qu'elle est économiquement très satisfaisante.\\

Avec quoi coupe t'on l'essence ?\\
Avec de l'éthanol.\\

D'où vient le bio-éthanol?\\
De la betterave en Europe (France particulièrement), et de la canne à sucre au brésil notamment.\\

Quels sont les défauts des principaux solvants ?\\
Ils sont très volatils, et souvent toxiques.\\

Quels sont les propriétés de l'éther ($H_5C_2OC_2H_5$) ?\\
polaire et aprotique.\\

Comment connaître la durée de passage au micro-ondes pour la manipulations de fin ?\\

Quel catalyseur avez vous utilisé ?\\
Acide paratoluasulfonique ($HO_3S-$toluène)\\

Quel est la formule du Toluène ?\\
C'est du benzène (cycle à 6) + un méthyl.\\

Quels sont les oxydants usuels en chimie organique ?\\
Le permenganate, le bicromate ($Cr_{2}O_7^{2-}$), le dioxygène, l'eau oxygénée.\\

Que signifie bio-sourcé ? Cela équivaut il à biodégradable ?\\
Cela signifie que c'est formé a partir de matériaux végétaux, mais n'est pas nécessairement biodégradable.

\section*{Remarques} 
Il faut souligner le fait que la chimie organique a pour matériaux de base des dérivés pétrolier.\\
Il faut tacher de rester politiquement neutre dans cette leçon.\\
Possibilité de donner un exemple plus ambitieux comme celui de l'ibuprofène pour illustrer l'utilisation atomique.\\
On peut donner des exemples plus spécifiques et plus complexes : on doit montrer que l'on a des connaissances en chimie.\\
Différencier valorisation énergétique et chimique.\\
Semble plus pédagogique et dynamique de partir de manip (un plastique (ou un biocarburant) et une synthèse) et d'introduire les principes en les commentant. On peut alors éventuellement citer les principes que l'on a pas utilisés en ouverture, mai ce n'est pas une obligation ?\\
Pour les programmes de lycée regarder "apport de la chimie à l'environnement".



\end{document}