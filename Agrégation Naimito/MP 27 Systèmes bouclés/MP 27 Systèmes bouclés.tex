\documentclass[12pt,prb,aps,epsf]{report}
\usepackage[utf8]{inputenc}
\usepackage{amsmath}
\usepackage{amsfonts}
\usepackage{amssymb}
\usepackage{graphicx} 
\usepackage{latexsym} 
\usepackage[toc,page]{appendix}
\usepackage{listings}
\usepackage{xcolor}
\usepackage{soul}
\usepackage[T1]{fontenc}
\usepackage{amsthm}
\usepackage{mathtools}
\usepackage{setspace}
\usepackage{array,multirow,makecell}
\usepackage{geometry}
\usepackage{textcomp}
\usepackage{float}
%\usepackage{siunitx}
\usepackage{cancel}
%\usepackage{tikz}
%\usetikzlibrary{calc, shapes, backgrounds, arrows, decorations.pathmorphing, positioning, fit, petri, tikzmark}
\usepackage{here}
\usepackage{titlesec}
%\usepackage{bm}
\usepackage{bbold}

\geometry{hmargin=2cm,vmargin=2cm}

\begin{document}
	
	\title{MP 27 Systèmes bouclés}
		\author{Maria}
	
	\maketitle
	
	\tableofcontents
	
	\pagebreak
\subsubsection{Introduction}
Définition d'un système bouclé. Classes de systèmes bouclés :

Stables : asservissements

Instables
\section{Oscillateur à pont de Wien}
\paragraph{Schéma de l'oscillateur} : schéma avec composantes, principe de fonctionnement, 
\subsubsection{Etude du filtre}
\begin{eqnarray}
H(j\omega) = \frac{1}{3+j\left(\frac{\omega}{\omega_0}-\frac{\omega_0}{\omega}\right)}=\frac{K_0}{1+jQ\left(\frac{\omega}{\omega_0}-\frac{\omega_0}{\omega}\right)}\\
Q = \frac{1}{3} \; \& \; K_0=\frac{1}{3}
\end{eqnarray}
\subsubsection{Fréquence de coupure et facteur de qualité du philtre}
On mesure les deux fréquences de coupure et la fréquence de résonance du filtre sur le diagramme de Bode établit en préparation.

\subsubsection{Étude du système bouclé}
On a une équation différentielle qui régit notre circuit
\begin{eqnarray}
\frac{dV_s^2}{dt^2} + \frac{3-G}{RC}\frac{dV_s}{dt} + \frac{V_s}{(RC)^2} = 0
\end{eqnarray}
On discute les trois cas :

$G<3$ $V_s$ tend vers 0

$G=3$ on a alors un oscillateur harmonique

$G>3$ $V_s$ tend alors vers l'infini.
\subsubsection{Pureté des oscillations}
On mesure ici le taux de distorsion à l'aide d'un distorsiomètre (qui a seulement $10k\Omega$ d'impédance : il faut dont placer un suiveur devant)
\begin{eqnarray}
\delta = \frac{\sqrt{\sum_{k=2}^{\infty} A_k^2}}{\sqrt{\sum_{k=1}^{\infty}A_k^2}} = 0.2\%
\end{eqnarray}

\section{Boucle à verrouillage de phase}
Schéma du circuit + principe du montage.
	\subsubsection{Caractérisation de l'OCT}
	On trace $f_s(V_{OCT}) = f_0-K_{OCT}V_{OCT}$ lorsque le signal est "accroché" afin de déterminer $K_{OCT}$. ($f_0$ est réglée sur l'OCT (2e GBF)).
	\subsubsection{Plage de verrouillage / plage de capture}
	Dans cette partie on va tracer $f_s(f_e)$ pour montrer dans quelle plage de fréquence le signal va accrocher, c'est à dire pour quelles valeurs de $f_e$ on va effectivement avoir $f_s=f_e$. On peut ainsi déterminer une plage de fréquence $\Delta f_c$ dans laquelle on commence à accrocher si ce n'était pas le cas avant, et une autre plage $\Delta f_v > \Delta f_c$ dans laquelle on continue à accrochait si on accrochait avant.\\
	Dans la théorie on attend :
	
	$\Delta f_v = K_{OCT}K_m V_e V_s$
	
	$\Delta f_c \simeq 2\sqrt{\Delta f_v f_e}$
	
	\subsubsection{Déphasage}
	Ici on trace $V_{OCT}(\Delta\phi)$ afin de déterminer $K_D$.
	
	\subsubsection{Stabilité}
	On discute ici la stabilité du système à partir de son gain en boucle ouverte qui permet d'obtenir "simplement" le gain général du système.
	
	\subsubsection{Rapidité}
	On envoie ici un échelon de fréquence dans la plage d'accroque avec le premier GBF et on mesure le temps que met la PLL pour se caler sur la nouvelle fréquence.
	
\section*{Questions}
Quelles sont les applications possibles du pont de Wien ?\\
Construire un GBF.\\

Si on utilise un pont de Wien pour un GBF, peut on choisir la fréquence du signal de sortie ? Peut on choisir la forme du signal produit ?\\
Oui en modifiant la fréquence de résonance du filtre, et non.\\

Pourquoi recherche t-on un signal pur ? Et qu'est ce que la pureté ici ?\\
On veut obtenir une sinusoïde idéale. Un signal pur est défini ici comme un signal ne possédant qu'une seule fréquence (dans la pratique cela équivaut à ce que l'amplitude des harmoniques soit négligeable devant celle de la fondamentale).\\

Pouvez vous nous montrer en direct si le signal que vous obtenez est bel et bien pur ?\\
On peut utiliser la fonction FFT de l'oscilloscope ou alors utiliser un logiciel tel que LatisPro après avoir enregistré le signal via une carte d'acquisition(attention dans ce cas à l'adaptation d'impédance : il va falloir modifier un peu R pour observer quelque chose), pour vérifier que les harmoniques sont bel et bien de très faible amplitude comparé à la fondamentale.

Pour la PLL, pourquoi avoir présenté la précision avant la stabilité ? Cela a t-il une importance ?\\

\section*{Remarques}
Le jury doit pouvoir lire en direct toutes les mesures.\\
Il faut essayer de faire le plus face possible au jury.\\
Le Wien sert dans la pratique à avoir un taux de distorsion record, le signal est plus pur que pour le meilleur GBF. Pour les Wien industriel le facteur de qualité est de l'ordre de $10^5$, et le gain est adaptatif afin de ne pas diverger. Ces oscillateurs sont utilisés par exemple pour la génération de porteuse en AM, ou pour tester le matériel hifi.\\
Il faut parler de la condition de "Barkhausen" : $|A||B|=1$.\\
On peut montrer la condition de démarrage, et le portrait de phase, qui est distordu si le taux de distorsion est non nul.\\
Il faut toujours avoir une carte d'acquisition dans le montage.\\
Si on a un problème d'accord en théorie et résultat exp : erreur de mesure, modèle inadapté, ou erreur de calcul.\\
Dans la PLL : bien accentuer le fait que l'ajustement entre les fréquences se fait sur la différence de phase.\\
Expliciter aussi que tous les composants sont ici linéaires, notamment pour la détermination de $K_D$.\\
On doit présenter la stabilité : si la différence de phase est toujours supérieure à $-\pi$ alors le système est inconditionnellement stable, avant la rapidité.\\
Précision : on regarde à quel point $f_s$ est similaire à $f_e$ : on visualise pour cela la tension $V_OCT$ qui serait un échelon pour un PLL idéal, et qui oscille dans la pratique : ile existe un régime transitoire. Si on diminue la Capacité C : on diminue la durée de ce régime transitoire mais on filtre moins, il y donc plus de bruit. \\
OCT = oscillateur contrôlé en tension (VCO en anglais)\\
Utilité de la PLL : pour reconstituer une porteuse, pour obtenir un signal très haute fréquence : on met en entrée un quartz par exemple, puis on utilise la PLL pour avoir $f_s = nf_e$. 
	

\end{document}