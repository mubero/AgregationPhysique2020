\documentclass[12pt,prb,aps,epsf]{article}
\usepackage[utf8]{inputenc}
\usepackage{amsmath}
\usepackage{amsfonts}
\usepackage{amssymb}
\usepackage{graphicx} 
\usepackage{latexsym} 
\usepackage[toc,page]{appendix}
%\usepackage{listings}
\usepackage{xcolor}
\usepackage{soul}
\usepackage[T1]{fontenc}
\usepackage{amsthm}
\usepackage{mathtools}
\usepackage{setspace}
\usepackage{array,multirow,makecell}
\usepackage{geometry}
\usepackage{textcomp}
\usepackage{float}
\usepackage{cancel}
\usepackage{here}
\usepackage{titlesec}
\usepackage{bbold}

\geometry{hmargin=2cm,vmargin=2cm}

\begin{document}
	
	\title{LP 27 Propagation guidée des ondes}
	\author{Naïmo Davier}
	\date{Agrégation 2019}
	\maketitle
	
	\tableofcontents
	
	\pagebreak
	
	
\section{Fibre optique}
\textbf{Optique physique} de \textit{Richard Taillet} aux éditions de Boeck (Noir et bleu) p 259.

\subsection{Guide géométrique simple}
Condition de réflexion totale : l'onde est alors confinée dans la fibre.

\subsection{Modes de transmissions}
Condition d'interférences constructives $\Longrightarrow$ notion de modes.

\subsection{Dispersion et bande passante}
On en déduit que la fibre est dispersante, ce qui induit une limitation quand à la fréquence maximale que l'on peut transmettre.


\section{Ondes électromagnétiques entre deux conducteurs plans}
\textbf{Electromagnétisme} de \textit{Pérez} p 594\\
\textbf{Electromagnétisme 2} de \textit{Faroux et Renault} 

\subsection{Calcul des champs}
On pose les bases, on explicite les conditions de passage pour l'interface vide-conducteur qui constituent ici nos conditions aux bords.

\subsection{Dispersion du mode TE}
Regarder directement le cas TE en disant que dans le cas pratique du guide rectangulaire, il n'existe pas de TEM.\\

Raccorder les résultats à la partie I.\\

Pour avoir un mode TEM dans un guide il faut donc placer un conducteur au centre qui autorise une différence de potentiel entre lui et le guide : câble coaxial.

\section{Câble coaxial}
\textbf{Electromagnétisme} de \textit{Pérez}\\;
\textbf{Onde électromagnétiques dans le vide et les milieux conducteurs} de \textit{C Garing} p39.\\

Expliquer que l'on peut réaliser une modélisation simple. Donner les résultats (selon le temps).


\section*{Questions}
Quelle est la définition de transverse électrique ?\\

Pour une propagation libre, que peut on dire de $E_z$ et $B_z$ ?\\
Ils sont nuls $\rightarrow$ dans le vide : TEM.\\

Pourquoi ne peut il y avoir de propagation dans un guide d'onde conducteur ?\\

Quelles sont les conditions limites vérifiées par le champ ?\\

Les ondes électromagnétiques sont elles les seules à être guidée ?\\

Fréquence de coupure d'une flûte ?\\

\section*{Remarques}
Le message de cette leçon est de faire apparaître la notion de guidage d'une onde.\\
On prend l'équation de propagation comme pré-requis.\\
C'est un problème de conditions limites.\\
Il faut mettre en valeur l'intérêt des guides d'ondes par rapport aux antennes : on limite la puissance émise, on la canalise uniquement vers les points d'intérêts. \\
Choix des ondes électromagnétiques pertinentes, mais les conditions au limite sont plus compliquées.\\
On peut aussi traiter les équations acoustiques, où d'Alembert est alors scalaire, et oùles conditions aux limites sont plus faibles.\\
Il faut discuter le signe des constantes que l'on établit.\\
Parler d'ondes évanescentes.\\
Pour les conditions limites : Il faut donner les relations de passage !\\

$E_t$ continu.

$[E_n] = \frac{\sigma}{\varepsilon_0}$

et donner aussi pour B.

	
\end{document}