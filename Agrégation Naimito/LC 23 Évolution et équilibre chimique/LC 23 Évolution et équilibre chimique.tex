\documentclass[12pt,prb,aps,epsf]{article}
\usepackage[utf8]{inputenc}
\usepackage{amsmath}
\usepackage{amsfonts}
\usepackage{amssymb}
\usepackage{graphicx} 
\usepackage{latexsym} 
\usepackage[toc,page]{appendix}
%\usepackage{listings}
\usepackage{xcolor}
\usepackage{soul}
\usepackage[T1]{fontenc}
\usepackage{amsthm}
\usepackage{mathtools}
\usepackage{setspace}
\usepackage{array,multirow,makecell}
\usepackage{geometry}
\usepackage{textcomp}
\usepackage{float}
\usepackage{cancel}
\usepackage{here}
\usepackage{titlesec}
\usepackage{bbold}

\geometry{hmargin=2cm,vmargin=2cm}

\begin{document}
	
	\title{LC 23 Évolution et équilibre chimique}
	\author{Mathieu}
	\date{Agrégation 2019}

	\maketitle
	
	\tableofcontents
	
	\pagebreak

\subsection*{Pré-requis}
Niveau deuxième année.

\section{Expériences introductives}
\subsection{Dissociation de l'acide formique}
\paragraph{Manip :} \textit{Terminale S Hachette édition 2012} p 328. \\

On fait un tableau d'avancement pour expliciter la réaction de l'acide méthanoïque $HCOOH$ sur l'eau. On fait un suivi conductimétrique de l'avancement car les produits sont des ions (on connaît $\lambda_{H_3O^+}$ et $\lambda_{HCOO^-}$). On en déduit les concentrations à l'équilibre, en fonctions de différentes concentrations initiales d'acides (on fait différentes dilutions de la solution mère). A partir de ces valeurs on va pouvoir montrer que la constante d'équilibre $K_a$ est bel et bien une constante (qui ne dépend pas des proportions initiales).\\ Il faut prendre la précaution de faire toutes les mesures à la même température, $K_a$ et les $\lambda$ dépendant de T à priori, ce que l'on fait au moyen d'un bécher thermostaté à 25°C (température pour laquelle on connait les conductivités molaires).\\
On en déduit que pour une température donnée, l'état d'équilibre atteint sera toujours le même, quelque soient les proportions initiales de réactifs (l'avancement sera donc lui différent).

\subsection{Dimérisation du dioxyde d'azote}
\paragraph{Manip :} On a la réaction
\begin{eqnarray}
2NO_{2(g)} = N_2O_{4(g)}
\end{eqnarray}
qui est exothermique, où les deux composés n'ont pas la même couleur : $NO_{2(g)}$ est incolore tandis que $N_2O_{4(g)}$ est brunâtre.\\
On place originellement le gaz dans un volume entouré de glace. On le place ensuite à température ambiante et on voit que cela s'éclaircit.\\
On démontre ainsi que la constante de réaction dépend de la température, et qu'imposer une variation de cette dernière permet donc de déplacer l'équilibre.

\section{Interprétation thermodynamique}
\subsection{Position du problème}
Si on reprend le premier exemple introduit, on a une réaction de la forme 
\begin{eqnarray}
0 = \sum _i \nu_i A_i
\end{eqnarray}
où les $\nu_i$ sont négatifs pour les réactifs et positifs pour les produits.\\
On peut introduire l'enthalpie libre 
\begin{eqnarray}
G(T,P,\{n_i\})\\
dG = \sum_i \left(\frac{\partial G}{\partial n_i}\right)_{T,P,\{n_{autres}\}}dn_i\\
or\; n_i = n_{i}^0 + \nu_i \xi\\
dG = \sum_i \nu_i \left(\frac{\partial G}{\partial n_i}\right)_{T,P,\{n_{autres}\}}d\xi = \sum_i \nu_i \mu_i(T,P)d\xi\\
\Delta_rG = \left(\frac{\partial G}{\partial \xi}\right)_{T,P} = \sum \nu_i\mu_i \;\rightarrow \;dG = \Delta_rG d\xi\\
\mathrm{On\;a}\;\mu_i(T,P) = \mu_i^o(T)  +RT\ln a_i\\
\mathrm{D'ou}\; \Delta _rG = \Delta _rG^o + \sum_i \nu_i RT\ln a_i = \Delta _rG^o + RT \ln \Pi a_i^{\nu_i}\\
= \Delta_rG^o + RT \ln\mathcal{Q}_r
\end{eqnarray}
\subsection{Critère d'évolution spontanée}
\begin{eqnarray}
dU = \delta W + \delta Q\\
dS = \frac{\delta Q}{T_{ext}} + \delta_i S\\
G = U + PV - TS\;\Rightarrow\; dG = -T_{ext}\delta _i S 
\end{eqnarray}
a T, P et S constantes, d'où $dG \le 0$.

\subsection{Prévision de l'évolution et équilibre}
\begin{eqnarray}
\Delta _rG d\xi = -T\delta_i S \; \Rightarrow \; \delta_i S = -\frac{\Delta_rG d\xi}{T}\ge 0
\end{eqnarray}
On a
\begin{eqnarray}
\Delta_rG = \Delta_rG^o + RT \ln\mathcal{Q}_r
\end{eqnarray}
or $\Delta_rG = 0$ à l'équilibre d'où 
\begin{eqnarray}
\Delta _rG^o(T) = -RT \ln \mathcal{Q}_r^{eq} \\
\Rightarrow\;\Delta_rG = RT \ln \frac{\mathcal{Q}}{K^o}
\end{eqnarray}
Le signe de $\Delta _rG$ nous informe sur le sens d'évolution de la réaction car $\Delta_rG d\xi \le 0$, on en déduit que l'on peut prévoir le sens d'avancement de la réaction en comparant $\mathcal{Q}_r$ et $K^o$.

\section{Influence de perturbations sur l'équilibre}
\subsection{Notion de Variance}
\subsection{Influence de la température}
\subsection{Influence de la pression}

\section*{Questions}
Concernant la dimérisation 
\begin{eqnarray}
2NO_{2(g)} = N_2O_{4(g)}
\end{eqnarray}
quels sont les noms des produits en jeu ?\\
On ne peut pas dire peroxyde car il n'y a pas de liaison O-O. On dit donc dioxyde d'azote et tetraoxyde de diazote.\\

Pouvez vous calculer la variance dans ce cas ?\\
On a comme variables $T,\,P,\,x_{NO_2},\,x_{N_2O_4}$. On a deux relations : $\sum x_i = 1$ et l'expression de $K^o$. On a donc $v=2$.\\

Qu'est ce qu'une rupture d'équilibre ?\\

Lorsqu'on fait la deuxième manip, comment contrôle t-on la pression ?\\

Quel est l'impact de la pression sur cette réaction ?\\

Dans la première manip, peut on montrer la dépendance de $K_a$ en la température ?\\
Le problème va être pour les conductivités molaires, qui ne sont en général connue qu'à $25^oC$.

\section*{Remarques}
Il ne faut pas faire la dilution en direct : trop long.\\
Concernant la dissociation de l'acide : il faut calculer les rendements pour les deux $C_i$ initiales, et commenter : introduit la loi de Dilution d'Ostwald.\\
Attention pour les $\lambda^0$ ce sont les coeff à dilution infinie et non dans un état standard.\\
Attention : le fait de ne pas atteindre l'équilibre (rupture d'équilibre) si l'on ajoute trop peu de réactif ne peut pas arriver si on a que des phases homogènes. L'exemple pertinent est le fait que le Ks n'est pas respecté pour une précipitation tant que l'on est pas à saturation.


\end{document}