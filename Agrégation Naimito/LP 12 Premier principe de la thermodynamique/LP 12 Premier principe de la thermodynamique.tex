\documentclass[12pt,prb,aps,epsf]{report}
\usepackage[utf8]{inputenc}
\usepackage{amsmath}
\usepackage{amsfonts}
\usepackage{amssymb}
\usepackage{graphicx} 
\usepackage{latexsym} 
\usepackage[toc,page]{appendix}
\usepackage{listings}
\usepackage{xcolor}
\usepackage{soul}
\usepackage[T1]{fontenc}
\usepackage{amsthm}
\usepackage{mathtools}
\usepackage{setspace}
\usepackage{array,multirow,makecell}
\usepackage{geometry}
\usepackage{textcomp}
\usepackage{float}
%\usepackage{siunitx}
\usepackage{cancel}
%\usepackage{tikz}
%\usetikzlibrary{calc, shapes, backgrounds, arrows, decorations.pathmorphing, positioning, fit, petri, tikzmark}
\usepackage{here}
\usepackage{titlesec}
%\usepackage{bm}
\usepackage{bbold}

\geometry{hmargin=2cm,vmargin=2cm}

\begin{document}
	
	\title{LP 12 Premier principe de la thermodynamique}
		\author{Etienne}
	
	\maketitle
	
	\tableofcontents
	
	\pagebreak
	
\subsubsection{Pré-requis}
Niveau MPSI 1ere année\\
Forces\\
variables d'états
\subsubsection{Introduction}
Caractère intuitif de la conservation de l'énergie, mais principe faisant toujours débat aujourd'hui. On se place en thermodynamique, et on sera donc ici dans des systèmes thermodynamiques, à l'équilibre, et donc décrit par quelques variables.\\

\section{De la mécanique à la thermodynamique}
\subsection{Energie d'un système thermodynamique}
On définit un système $\mathcal{S}$, contenant N particules de masse m, et dont le centre de masse sera noté G.
\begin{eqnarray}
\varepsilon_c = \varepsilon_{macro} + \varepsilon_{micro} = \frac{1}{2}NmV_G^2 + \sum\frac{1}{2}mv_i^{*2}\\\varepsilon_p = \varepsilon_{p\,int} + \varepsilon_{p\,ext}\\
= \varepsilon_{agit} + \varepsilon_M
\end{eqnarray}
Pour un système on définit donc l'énergie interne comme
\begin{eqnarray}
	U= \varepsilon_{agit} + \varepsilon_{p\,int},
\end{eqnarray}
c'est une fonction d'état, on va voir si elle est extensive.\\
Exemple d'un mélange eau-huile

\subsection{Transfert d'énergie sous forme de travail}
On considère un gaz dans un volume V, fermé d'un coté par un piston...\\
Il existe de nombreux types de travail : mécanique, de couple, électrique, magnétique, chimique (on donnera leurs expressions)..etc
\subsection{Une autre forme de transfert d'énergie ? La chaleur}
Exemple d'une bouilloire.\\
Ici on va considérer en réalité des valeurs moyennes, puisque les valeurs microscopique fluctuent dans le temps et dans l'espace à des échelles que l'on ne peut percevoir avec des outils "courants".
\section{Premier principe}
\subsection{Énoncé}
U fonction d'état.
\begin{eqnarray}
\Delta U = W+Q\\dU = \delta W + \delta Q
\end{eqnarray}
on note dU et $\delta W$ car U est une fonction définie à chaque état, et ne dépend pas du chemin suivi, on peut donc définir sa différentielle. Tandis que Q et W dépendent du chemin suivi.
\subsection{Définition de la chaleur}
Exemple de la terre autour du soleil.
Exemple d'un système diatherme.
\section{Transfert calorifique}
\subsection{Capacités calorifiques}
\begin{eqnarray}
U(T,V)\\dU = \left(\frac{\partial U}{\partial T}\right)_VdT + \left(\frac{\partial U}{\partial V}\right)_TdV\\
\delta Q = dU - \delta W\\
\delta Q = \left(\frac{\partial U}{\partial T}\right)_VdT + \left[ \left(\frac{\partial U}{\partial V}\right)_T-P_0\right]dV\\
C_v = \left(\frac{\partial U}{\partial T}\right)_V
\end{eqnarray}
avec $C_v$ en $J.K^{-1}$.
\begin{eqnarray}
\Delta U = Q_V = \int_{T_1}^{T_2}C_v(T,V)dT
\end{eqnarray}
Remarque : on peut aussi travailler avec l'enthalpie H, et définir $C_p= \left(\frac{\partial H}{\partial T}\right)_P$.

\subsection{Application}
Exemple d'un café : on veut sucrer(avec 10g de sucre à 20°C) un café à $50°C$ tout en conservant notre café à la même température.\\
Seule la gravité travaille ici
\begin{eqnarray}
W = mgh\\
\Delta U = \Delta U_{cafe} + \Delta U_{sucre} = 0 + mC_m(T_F-T_I)\\
\Rightarrow h= \frac{C_m(T_F-T_I)}{g} = 1529m!!!
\end{eqnarray}
On peut aussi lancer le sucre avec une vitesse initiale de 622km/h, ou bien prendre un température initiale de 50,75°C.\\
Permet de mettre en évidence la notion de transfert thermique.\\
Commentaire quand aux capacités calorifiques de quelques métaux, de l'air et de l'eau dans ses trois états directement.
\subsection{Calorimétrie}
Explication du concept expérimental permettant de mesurer ces capacités calorifiques.\\
La capacité calorifique dépend du nombre de degrés de liberté.

\section*{Questions}
Pour la calorimétrie : on travaille à pression constante ... sur les schéma on travaille sous vide .. qu'en est il ?\\
Comment fait on donc en pratique pour étudier la capacité calorifique ?\\

Concernant l'exemple de la tasse de café : comment obtient on le h ? Est ce réaliste ?\\

Si je jette un caillou dans une marre, est ce que le caillou chauffe la marre ?\\

Lorsque je tourne ma cuillère dans mon café, est ce que je chauffe mon café ?\\

Quand un enfant remue sa soupe elle refroidit plus vite... pourquoi ?\\

Pourquoi lorsqu'on souffle sur sa soupe elle refroidit encore plus vite ??\\

A t-on toujours le droit de décomposer l'énergie cinétique et énergie c macro et micro ?\\
Théorème de Koenig\\

Pourquoi ne peut on pas toujours définir un centre de masse invariant pendant une transformation ?\\

Qu'est ce qu'une fonction d'état ?\\

Comment définit on un état thermodynamique ?\\

Est on obligé d'être à ol'équilibre pour définir l'énergie interne ?\\

Un exemple où l'on peut chauffer sans augmenter la température ?\\
Un exemple ou Q=0 et où $\Delta T \neq 0$ ?\\
Pompe à vélo, aérosol.\\

Si on laisse une cocote sur le feu trop longtemps elle explose ... pourquoi ?\\ Si le gaz dans la cocote est parfait ... explose t-elle quand même ?\\

\section*{Remarques}
Leçon trop théorique : il faut plus d'exemples, et notamment plus d'exemples concrets.\\
Concernant le centre de masse : on caractérise un état particulier, à un instant donné, on se moque donc du centre de masse.\\
mauvaise définition d'état thermodynamique.\\
Insister sur le passage microscopique $\rightarrow$ macroscopique.\\
Définir la notion de transformation ou la mettre en pré-requis.\\
Pour les moyennages de la pression etc... faire attention à la rigueur des expressions données, par ex pour la pression on définit la force moyenne reçue par chaque particule, puis on divise par la surface.\\

Plan possible :\\

Donner un exemple concret en introduction : on établit une problématique qu'on résout pendant la leçon\\
Ex possible : on pose une masse sur le piston, on chauffe et la masse monte, ou bien exemple de la voiture qui freine.\\

On définit l'énergie interne, et aussi la notion d'équilibre.\\

Il faut revenir à un moment à l'exemple initial.





\end{document}
