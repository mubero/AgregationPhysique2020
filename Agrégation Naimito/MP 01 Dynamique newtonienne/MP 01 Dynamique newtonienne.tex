\documentclass[12pt,prb,aps,epsf]{article}
\usepackage[utf8]{inputenc}
\usepackage{amsmath}
\usepackage{amsfonts}
\usepackage{amssymb}
\usepackage{graphicx} 
\usepackage{latexsym} 
\usepackage[toc,page]{appendix}
%\usepackage{listings}
\usepackage{xcolor}
\usepackage{soul}
\usepackage[T1]{fontenc}
\usepackage{amsthm}
\usepackage{mathtools}
\usepackage{setspace}
\usepackage{array,multirow,makecell}
\usepackage{geometry}
\usepackage{textcomp}
\usepackage{float}
\usepackage{cancel}
\usepackage{here}
\usepackage{titlesec}
\usepackage{bbold}

\geometry{hmargin=2cm,vmargin=2cm}

\begin{document}
	
	\title{MP 01 Dynamique newtonienne}
	\author{Naïmo Davier}
	\date{Agrégation 2019}

	\maketitle
	
	\tableofcontents
	
	\pagebreak
	
	
\section{Principe fondamental de la dynamique}
On utilise le banc pasco.\\

\textbf{Remarque} : La composante tangentielle du frottement solide étant proportionnelle à la masse, on aura intérêt à ne pas charger les oitures pour avoir de meilleurs résultats.

\subsection{Mesure de $g$}
On incline le banc à l'aide d'un élévateur, et on mesure l'angle d'inclinaison $\alpha$ avec un outils composé d'un rapporteur et d'un niveau à bulle. On place alors la voiturette, déchargée du coté surélevé, et on la laisse descendre sous l'effet de la gravité. On mesure alors une accélération constante depuis l'interface : utiliser l'accélération basée sur la dérivée de la position (elle même mesurée grâce à la rotation des roues), il existe aussi un capteur d'accélération "absolue" qui prend déjà l'accélération de la pesanteur en compte : pas pertinent ici.\\

On connait $\alpha$ et l'accélération selon l'axe du banc $a$ : on en déduit alors 
\begin{eqnarray}
g = \frac{a}{\sin\alpha}
\end{eqnarray}
On peut estimer l'angle sur l'angle à partir de notre lecture, et on peut regarder si la valeur de l'accélération mesurée fluctue pour y associer une incertitude. On a alors, en notant $g = f(a, \alpha)$
\begin{eqnarray}
u(g)^2 =  \sum_i \left(\partial_if\;u(i)\right)^2 = \frac{u(a)^2}{\sin^2\alpha} + \frac{a^2u(\alpha)^2}{\tan^2\alpha\sin^2\alpha}\\
\Longrightarrow \left(\frac{u(g)}{g}\right)^2 = \left(\frac{u(a)}{a}\right)^2 + \left(\frac{u(\alpha)}{\tan\alpha}\right)^2
\end{eqnarray}

\subsection{Vérification du PFD}	
On conserve le banc incliné. On prend une voiturette que l'on charge, que l'on équipe d'un embout avec un crochet permettant de mesurer une force, et que l'on lie à un coté du banc à l'aide d'un ressort (prendre un "long mou" pour avoir des oscillations lentes). On peut alors mesurer la force sur le crochet en fonction du temps $F(t)$ et l'accélération selon l'axe du banc $a$(là encore liée à la position) et alors calculer dans l'onglet "calculatrice" 
\begin{eqnarray}
D(t) = ma(t) -F(t) + mg\sin\alpha
\end{eqnarray}
où la valeur du terme constant $mg\sin\alpha$ peut être calculée à partir de la valeur de $g$ précédemment déterminée. On a normalement $D(t) = 0\; \forall\, t$, dans la pratique $D$ oscille autour de zéro, on pourra donc regarder la valeur de 
\begin{eqnarray}
n =100\,\frac{\max(D)}{\max(F)}
\end{eqnarray}
On montre alors qu'on vérifie le PFD à $n$\%.
	
\section{Collision élastique}
On utilise là encore le banc. On installe les "pare-chocs aimantés" sur les voiturettes, on les charges différemment et on les fait entrer en collision. On vérifie alors que la quantité de mouvement totale 
\begin{eqnarray}
P = m_1v_1+m_2v_2
\end{eqnarray}
est conservée. Là encore la mesure n'est pas absolue, il faut regarder la valeur de
\begin{eqnarray}
n = 100\,\frac{\max(P) - \min(P)}{\langle P \rangle}
\end{eqnarray}	
et on a ainsi une vérification de la conservation de la quantité de mouvement lors d'une collision élastique à $n$\%.\\

On peut aussi se servir des scratch pour mimer un type de collision différente où les voiturettes vont rester collé après l'impact.

\section{Pendule pesant}
On a pendule fait d'une tige métallique de longueur $L$ cm et de masse $m$, fixée à l'une de ses extrémités à un potentiomètre. On y ajoute une masselotte $M$ à la distance $l$ du centre de rotation. On applique alors le théorème du moment cinétique 
\begin{eqnarray}
\frac{d\vec{L}}{dt} = \vec{M}
\end{eqnarray}
ou niveau du centre de rotation, avec comme seul moment celui du poids sur la tige et la masselotte. On a ainsi, en désignant par $\theta$ l'angle que la barre fait avec la verticale, le TMC qui s'écrit 
\begin{eqnarray}
(I_b + I_{m} )\dot{\theta} = \left(m\frac{L}{2} + Ml\right)g\sin \theta
\end{eqnarray}
si on se place pour de petites oscillations on a alors la pulsation qui s'écrit 
\begin{eqnarray}
\omega^2 = \frac{mL/2 + Ml}{ mL^2/3 + Ml^2}g \label{omega}
\end{eqnarray}
qui est une expression assez compliquée. On constate cependant que pour $M\gg m$ et $l \geq L/2$ on retrouve le cas du pendule simple 
\begin{eqnarray}
\omega^2 \simeq \frac{g}{l}
\end{eqnarray}
Donc on sait que lorsqu'on va tracer 
\begin{eqnarray}
\frac{\omega^2}{g} = f(l)
\end{eqnarray}
on s'attend à avoir une allure en $1/l$ lorsque $L/2 \leq l \leq L$.\\

Il faut, pour que régressi arrive à converger, bien couvrir l'intervalle $l\in [0, L]$, et on pourra alors vérifier la cohérence de la théorie en remontant à une valeur de $M$ en modélisant la courbe obtenue par la fonction (\ref{omega}), que l'on comparera à la valeur obtenue à l'aide d'une balance.


\end{document}