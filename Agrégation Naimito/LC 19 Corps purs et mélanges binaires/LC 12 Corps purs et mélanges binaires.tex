\documentclass[12pt,prb,aps,epsf]{article}
\usepackage[utf8]{inputenc}
\usepackage{amsmath}
\usepackage{amsfonts}
\usepackage{amssymb}
\usepackage{graphicx} 
\usepackage{latexsym} 
\usepackage[toc,page]{appendix}
\usepackage{listings}
\usepackage{xcolor}
\usepackage{soul}
\usepackage[T1]{fontenc}
\usepackage{amsthm}
\usepackage{mathtools}
\usepackage{setspace}
\usepackage{array,multirow,makecell}
\usepackage{geometry}
\usepackage{textcomp}
\usepackage{float}
%\usepackage{siunitx}
\usepackage{cancel}
%\usepackage{tikz}
%\usetikzlibrary{calc, shapes, backgrounds, arrows, decorations.pathmorphing, positioning, fit, petri, tikzmark}
\usepackage{here}
\usepackage{titlesec}
%\usepackage{bm}
\usepackage{bbold}

\geometry{hmargin=2cm,vmargin=2cm}

\begin{document}
	
	\title{LC 12 Corps purs et mélanges binaires}
	\author{Mathieu}
	\date{Agrégation 2019}
	
	\maketitle
	
	\tableofcontents
	
	\pagebreak
	
\section{Le potentiel chimique}
\section{Changement d'état du corps pur}
\section{Critère d'évolution}
\subsection{Diagramme de phase}
\paragraph{Manip :}  On plonge de l'acide acétique (=ac éthanoïque $CH_3COOH$, dont la température de fusion est très haute : $T_{fusion} = 16,6$°C) liquide dans ,un dewar rempli de glace et d'une solution saturée en sel (pouvant donc atteindre environ -20°C pour une solution à 23\% en sel) et on regarde la température de l'acide en fonction du temps : on constate alors un palier à $T=T_{fusion}$ lors de la cohabitation des phases liquide et solide.

\section{Mélanges binaires solide liquide}
\subsection{Potentiel chimique d'un mélange}
\subsection{Refroidissement d'un mélange}
\textbf{Manip :} On refait la même expérience mais avec cette fois un mélange d'eau et d'acide acétique avec une fraction massique d'eau de 10\%.

\subsection{Mélange binaire homogène}
\subsection{Mélange binaire inhomogène}
\subsubsection{Miscibilité partielle à l'état solide}

\section*{Questions}
Niveau de cette séquence et pré-requis ?\\
Deuxième année de prépa PSI. Premier et second principe.\\

Comment définissez vous la variance ?\\
Nombre de paramètres intensifs indépendants.\\

Pourquoi a t-on une variance de 2 dans le cas de l'eau liquide ?\\
On a 2 paramètres intensifs : P et T, et aucune relation les liants.\\

Comment aurait pu t-on faire la courbe de refroidissement de l'eau ?\\ 
Il faut abaisser la température de l'eau en dessous de 0°C, on peut le faire en utilisant une solution d'eau salée, dont la température de l'eutectique est d'environ -21°C.\\

A quelle condition la miscibilité est elle totale ?

\section*{Remarques}
Trop long : il faut enlever ou au moins beaucoup raccourcir la partie 1 et 2.1.\\
Il faut un plus contextualiser. Parler de soudure par exemple. Mélange racémique ayant une température de fusion différente des deux énantiomères.\\
Le changement d'état d'un corps pur est un rappel.\\
Il faut définir la variance.\\
On ne doit pas spoiler le résultat des expériences !\\


\end{document}