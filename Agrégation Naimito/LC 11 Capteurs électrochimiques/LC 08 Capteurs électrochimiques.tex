\documentclass[12pt,prb,aps,epsf]{article}
\usepackage[utf8]{inputenc}
\usepackage{amsmath}
\usepackage{amsfonts}
\usepackage{amssymb}
\usepackage{graphicx} 
\usepackage{latexsym} 
\usepackage[toc,page]{appendix}
\usepackage{listings}
\usepackage{xcolor}
\usepackage{soul}
\usepackage[T1]{fontenc}
\usepackage{amsthm}
\usepackage{mathtools}
\usepackage{setspace}
\usepackage{array,multirow,makecell}
\usepackage{geometry}
\usepackage{textcomp}
\usepackage{float}
%\usepackage{siunitx}
\usepackage{cancel}
%\usepackage{tikz}
%\usetikzlibrary{calc, shapes, backgrounds, arrows, decorations.pathmorphing, positioning, fit, petri, tikzmark}
\usepackage{here}
\usepackage{titlesec}
%\usepackage{bm}
\usepackage{bbold}
\geometry{hmargin=2cm,vmargin=2cm}

\begin{document}
	
	\title{LC 08 Capteurs électrochimiques}
		\author{Naïmo Davier}
		\date{Agrégation 2018}
	
	\maketitle
	
	\tableofcontents
	
	\pagebreak
	
\subsubsection{Pré-requis}

 Piles électrochimiques\\
Notion de base en électrocinétique\\
Dosages.
\subsubsection{Intro}

Définition de capteur, puis de capteur électrochimique.\\
Utilisation.

\section{Piles électrochimiques et potentiel d'électrode}
\subsection{Pile électrochimique}
Pile Daniel : schéma, plus principe de fonctionnement, avec illustration en direct : \textbf{Girard} \textit{Chimie inorganique et générale} p64 ou \textbf{Le maréchal} \textit{Chime générale} p190.\\
Explication concernant la mesure de courant circulant entre l'anode et la cathode (à définir) de la pile.\\
Que se passe t'il si on place une grosse résistance entre les deux bornes (cf si je place un voltmètre) : notion de courant sortant de l'anode pour aller vers la cathode à illustrer avec un voltmètre.
\subsection{Electrode et électrode de référence}
Force électromotrice dépendant de $\Delta E = E_{cathode} - E_{anode} = E_c-E_a$ $\Rightarrow$ nécessite une référence $E_0$ :
\begin{eqnarray}
\Delta E = E_c-E_0 - (E_a-E_0) = E_C-E_A
\end{eqnarray}
Présentation du concept d'électrode de référence avec l'électrode standard à hydrogène.
\paragraph{Présentation de l'électrode au calomel saturé} : schéma, fonctionnement, avantages (surtout très stable).

\section{Relation de Nernst}
\subsection{Mise en évidence expérimentale}
\textbf{Cachau Herreillat} \textit{Des expériences de la famille Red-OX} p227.\\

Montage : $AgNO_3$ en solution, électrode d'argent et électrode au calomel saturé afin de déterminer le potentiel du couple $Ag^+/Ag_{(s)}$. On trace 
\begin{eqnarray}
E = \Delta E_{mes} + E_{ECS} = f(\log_{10}[Ag^+])
\end{eqnarray}
Et on obtient ainsi une droite : $\Delta E = \alpha + \beta \log([Ag^+])$ avec $\alpha$ et $\beta$ en V.\\
On compare ensuite aux valeurs attendues.

\subsection{Enoncé de la loi de Nernst}
On considère la réaction 
\begin{eqnarray}
a\,Ox + b\,B + ne^- = C\,Red + dD
\end{eqnarray}
on a alors la loi de Nernst qui s'exprime comme
\begin{eqnarray}
\Delta E = E^0(Ox/Red,T) + \frac{RT\ln(10)}{nF}\log\left(\frac{[Ox]^{a}[B]^b}{[Red]^c[D]^d}\right)
\end{eqnarray}
On peut illustrer avec quelques exemples : \textit{ne faire que le premier (pas le temps)}\\

 l'électrode d'argent correspondant au couple $Ag^+/Ag_{(s)}$ : $Ag^+ + e^- = Ag_{(s)}$. (remarque : permet de mesurer une concentration en ions chlorure $AgCl_{(s)} + e^- = Ag_{(s)} + Cl^-$).
 
 ESH correspondant au couple $ H^+/H_2{(g)}$ : $2H^+ + 2e^- = H_{2(g)}$.
 
 Électrode au calomel saturé ECS : couple $Hg_2Cl_{2(s)}/Hg_{(l)}$ : $Hg_2Cl_{2(s)} + 2e^- = 2Hg_{(l)} + 2Cl^-$.
 
 \subsection{Polarité et force électromotrice d'une pile}
 
Dans le cas d'une pile (schéma) on peut écrire la relation de Nernst pour les deux électrodes, à la cathode et à l'anode. On regarde ici en direct ce qu'il se passe pour une pile daniel où les couples en jeu sont donc $Cu^+/Cu_{(s)}$ et $Zn^{2+}/Zn_{(s)}$.\\
Remarque sur la dépendance au PH du potentiel d'électrode en cas de présence d'ions $H_3O^+$.

\section{Capteurs électrochimiques}
On va commencer par parler de la sélectivité d'un capteur.
\subsection{Dosage potentiométrique - Sélectivité}
\subsection{Mesure du PH : électrode de Verre}
\subsection{Mesure de la teneur en $O_2$ de gaz d'échappement : sonde lambda}


\section*{Questions}
Quelle est la définition d'une électrode ?\\

Pour la relation de Nernst, que met on lorsque l'un des réactifs est solide ou gazeux ?\\

Dans la demie équation de l'ECS : pourquoi fait on apparaître des ions $Cl^-$ ?\\
Pouvez vous écrire les demies équations Red-Ox en jeu ?\\
\begin{eqnarray}
Hg_2Cl_{2(s)} = 2H_g^+ + 2Cl^-\\2e^- + 2Hg^+ = 2Hg_{(l)}\\
Hg_2Cl_{2(s)} + 2e^- = 2Hg_{(l)} + 2Cl^-
\end{eqnarray}

La notion de saturation (abordée) est elle connue des élèves de SPCL ?\\

Comment motiver dans la pratique (par un exemple pratique) les étudiants quand aux mesures par étalonnage présentées ici ?\\

Comment auriez vous fait pour déterminer la concentration d'une solution inconnue avec votre droite d'étalonnage ? Pouvez vous le faire ?\\

Existe il des capteurs électrochimiques qui mesurent autre chose qu'une différence de potentiel ?\\

Quel est le couple Red-Ox impliqué dans une électrode verre ?\\
Il n'y a pas de couple Red-Ox : électrode à membrane.

\section*{Remarques}
Tout prévoir à portée de main : pipette, propipettes, eau distillée (remplie) etc...\\
Boucher et secoucer les fioles jaugées pour homogénéiser !!\\
Sur les schémas attention à la représentation des ponts salins.\\
Mieux dégager sa paillasse pour que l'on puisse tout voir des manipulations.\\
Il faut équilibrer en direct et à la main.\\
Pour la loi de Nernst : introduire la notion d'activité, sans nécessairement formuler explicitement le concept : mais la différence doit être claire entre solide, liquide et gaz.\\
Essayer de passer plus vite sur la première partie.
\end{document}