\documentclass[12pt,prb,aps,epsf]{article}
\usepackage[utf8]{inputenc}
\usepackage{amsmath}
\usepackage{amsfonts}
\usepackage{amssymb}
\usepackage{graphicx} 
\usepackage{latexsym} 
\usepackage[toc,page]{appendix}
\usepackage{listings}
\usepackage{xcolor}
\usepackage{soul}
\usepackage[T1]{fontenc}
\usepackage{amsthm}
\usepackage{mathtools}
\usepackage{setspace}
\usepackage{array,multirow,makecell}
\usepackage{geometry}
\usepackage{textcomp}
\usepackage{float}
%\usepackage{siunitx}
\usepackage{cancel}
%\usepackage{tikz}
%\usetikzlibrary{calc, shapes, backgrounds, arrows, decorations.pathmorphing, positioning, fit, petri, tikzmark}
\usepackage{here}
\usepackage{titlesec}
%\usepackage{bm}
\usepackage{bbold}

\geometry{hmargin=2cm,vmargin=2cm}

\begin{document}
	
	\title{LP 21 Induction électromagnétique}
	\author{Hugo}
	\date{Agrégation 2019}
	
	\maketitle
	
	\tableofcontents
	
	\pagebreak
	
\subsection*{Prérequis}
Equations de Maxwell\\
Magnéto et électro-statique

\subsection{Introduction}
petite intro historique plus expérience : création d'un courant induit avec une bobine et un aimant.\\
Énoncé de la loi de Faraday + illustration avec expérience similaire.\\
Ce cours est notamment l'occasion de donner des applications de l'électrostatique.
	
\section{Force électromagnétique (f.e.m)}

\subsection{Cadre de l'étude}
On se place dans l'ARQS. 
\begin{eqnarray}
\vec{\nabla}\vec{E} = \vec{0}\\
\vec{\nabla}\times\vec{E} = -\frac{\vec{\partial\vec{B}}}{\partial t}\\
\vec{\nabla}\vec{B} = \vec{0}\\
\vec{\nabla}\times\vec{B} = \mu_0\vec{j}
\end{eqnarray}
Expression des champs en fonction des potentiels.\\
On établit la puissance P qui communique la vitesse $v_q$ à la particule q.
\begin{eqnarray}
P = \int_{\mathcal{C}} \frac{\vec{F}.\vec{dl}}{q}I\\
e = \frac{P}{q}
\end{eqnarray}
\subsection{Champ électromoteur}
On peut exprimer e en fonction de deux termes 
\begin{eqnarray}
e &=& \int_{\mathcal{C}} \vec{E}.\vec{dl} + \int_{\mathcal{C}} \vec{v}_q\times\vec{B}.\vec{dl} = - \int_{\mathcal{C}} \frac{\partial\vec{A}}{\partial t}.\vec{dl} + \int_{\mathcal{C}} \vec{v}_q\times\vec{B}.\vec{dl} \\
&=& \int_{\mathcal{C}} \vec{E}_m.\vec{dl}
\end{eqnarray}

\section{Induction de Lorentz}
\subsection{f.e.m}
On considèrera ici uniquement des circuits filiformes.\\
$\vec{B}$ uniforme donc la dérivée de $\vec{A}$ par rapport au temps est nulle. On en déduit 
\begin{eqnarray}
e = - \frac{d\Phi}{dt}
\end{eqnarray}

\subsection{Couplage électromagnétique - rail de Laplace}

Présentation de l'expérience du rail de Laplace et illustration par une manip en direct.\\
Calcul de $\vec{E}_m$ et ensuite de e. On peut aussi calculer $\Phi$ et en déduire à nouveau $e=-Blv$.\\
Explications et analogie avec le moteur à courant continu.

\section{Induction de Neumann}

\subsection{f.e.m}
on a ici $v\times B = 0$. On en déduit :
\begin{eqnarray}
e = \int_{\mathcal{C}} \frac{-\partial\vec{A}}{\partial t}.\vec{dl} = ... = -\frac{d\Phi}{dt}
\end{eqnarray}
avec $\Phi$ le flux de B à travers la surface associée à $\mathcal{C}$.

\subsection{Courants de Foucault}
$\Phi = BS = B_0\cos(\omega t)\pi r^2$ $\Rightarrow\; e = \frac{-d\Phi}{dt} = B_0\pi r^2 \omega \sin(\omega t)$.\\
Pour une spire on trouve après calcul l'expression de j : $j=\frac{B_0}{2}\gamma r \omega  \sin(\omega t)$.\\
Ordre de grandeur pour le cuivre $\frac{dT}{dt} \simeq 10K.s^{-1}$.\\
Analogie avce les plaques à induction.

\section{Coefficients d'inductance}
\subsection{Auto-induction}

Explication du phénomène et petit schéma de principe.\\
Cas d'un circuit isolé : $\vec{B} = 0$ et existence d'un courant I. On a 
\begin{eqnarray}
\Phi_p = \int\int_{S} \vec{B}_p.\vec{dS} = LI
\end{eqnarray}
Où L est l'inductance propre.
\subsection{Inductance mutuelle}
Cas où l'on a deux circuits :\\
$\Phi_p{12} = M_{12}I_1$\\
$\Phi_p{21} = M_{21}I_2$\\
$M_{21} = M{12} = M$ : coefficient d'induction mutuelle.\\
$e_1 = -\frac{d\Phi_{tot}}{dt}$ et $\Phi_{tot} = \Phi_p{p_1} + \Phi_p{p_2}$\\
On en déduit $e_1$ et $e_2$ puis $V_1$ et $V_2$, et ainsi, sachant que $L_i=L_0N_i^2$ et considérant un couplage idéal à savoir $M=\sqrt{L_1L_2}$ on peut calculer le rapport 
\begin{eqnarray}
\frac{V_1}{V_2} = \frac{N_2}{N_1}
\end{eqnarray}

\section*{Questions}

Pourquoi a t-on $M=\sqrt{L_1L_2}$ dans al dernière partie ?\\

Pourquoi a t-on $L_i=L_0N_i^2$ ?\\

Que se passe t-il lorsque l'on ajoute deux inductances en série ?\\
dépend du couplage magnétique : $L_{tot} = L_1 + L_2 + M$.\\

Quelle est l'énergie stockée dans une bobine ?\\

Y a t'il d'autres exemples d'application que ceux cités ?\\
Freinage par induction (ex : tramway).\\

Quelles sont les limites de ce freinage ? Peut il arrêter un véhicule seul ?\\
Très efficace à grande vitesse mais va devenir inefficace lorsque la vitesse va diminuer.\\

Pour les plaques à induction : quelle est la fréquence utilisée ?\\
Comment modifie t-on les 50 Hz du secteur pour utiliser les plaques du coup ?\\
En utilisant des composants non linéaires : ex AO en multiplieur, on multiplie le signal par lui même $\Rightarrow$ f $\rightarrow\;2f$.\\


Qu'est ce qui lie les inductions de Lorentz et de Neumann ?\\
Un changement de référentiel, qui va modifier les équations de $\vec{E}$ et $\vec{B}$.\\

Les potentiels des champs énoncés ici sont ils uniques ?\\
$\rightarrow$ transformation de jauge.\\

Qu'est ce que l'ARQS ? les équations établies ici sont elles alors toujours valables en dehors de ce régime ? \\

Quelle est l'orientation du $\vec{dS}$ pour une boucle donnée selon $\vec{dl}$ ? Et donc si e est négatif comment est i ?\\

Pour un contour donné, peut on prendre ensuite n'importe quelle autre surface ? Et si oui pourquoi ?\\
Car $\mathrm{div}\vec{B} = 0$ : B est à flux conservatif.\\

Lorsqu'on a un effet joule : quelle est la puissance volumique donnée aux charges ?\\
$P = \vec{j}.\vec{E}$.\\

Densité volumique des forces de Laplace ?

\section*{Remarques}
Attention on ne voit pas ici le lien avec Maxwell-Faraday :$\vec{\nabla}\times\vec{E} = - \frac{\partial\vec{B}}{\partial t}$.\\
Modifier le début : on arrive pas à bien voir la manip, introduction trop sophistiquée. ne pas donner les équations de maxwell en pré-requis car $\vec{\nabla}\times\vec{E} = - \frac{\partial\vec{B}}{\partial t}$ est le centre de la leçon : doit être commenté.\\
Il faut mettre la notion de flux $\phi$ plus en valeur, notamment pour la petite expérience.\\
On peut aussi partir de l'expérience pour établir $\vec{\nabla}\times\vec{E} = - \frac{\partial\vec{B}}{\partial t}$.\\
Discuter plus la physique.\\

Message essentiel de la leçon : l'induction c'est $\vec{\nabla}\times\vec{E} = - \frac{\partial\vec{B}}{\partial t}$ (ou $e=-\frac{d\Phi}{dt}$).\\

Il faut être capable d'expliquer en quelques phrases le choix de plan effectué.

\end{document}