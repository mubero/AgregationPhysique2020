\documentclass[12pt,prb,aps,epsf]{article}
\usepackage[utf8]{inputenc}
\usepackage{amsmath}
\usepackage{amsfonts}
\usepackage{amssymb}
\usepackage{graphicx} 
\usepackage{latexsym} 
\usepackage[toc,page]{appendix}
\usepackage{listings}
\usepackage{xcolor}
\usepackage{soul}
\usepackage[T1]{fontenc}
\usepackage{amsthm}
\usepackage{mathtools}
\usepackage{setspace}
\usepackage{array,multirow,makecell}
\usepackage{geometry}
\usepackage{textcomp}
\usepackage{float}
%\usepackage{siunitx}
\usepackage{cancel}
%\usepackage{tikz}
%\usetikzlibrary{calc, shapes, backgrounds, arrows, decorations.pathmorphing, positioning, fit, petri, tikzmark}
\usepackage{here}
\usepackage{titlesec}
%\usepackage{bm}
\usepackage{bbold}

\geometry{hmargin=2cm,vmargin=2cm}

\begin{document}
	
	\title{LP 26 Corrosion humide des métaux}
	\author{Naïmo Davier}
	
	\maketitle
	
	\tableofcontents
	
	\pagebreak
	
Prépa 2e année MP, PT
\section{Phénomène de corrosion}
Différents types de corrosions. Exemple du fer : demies équations dont on déduit l'équation de corrosion.\\

\textbf{Manip qualitative du clou dans l'agar agar avec un peu de phénolphtaléine pour mettre en évidence la corrosion} : \textit{ Cachau Herreillat redox p 166 et 187}\\

$\rightarrow$ cette manip pose différentes questions... On va maintenant y répondre.

\section{Corrosion uniforme : exemple du fer}
Def corrosion homogène.
\subsection{Approche thermodynamique / protection cathodique}
	Diagramme E(PH) du fer. Notion de zone d'immunité, zone de corrosion, zone de passivation (notion cinétique et non thermodynamique).
	
\subsection{Étude cinétique}
	On fait la manip : \textit{ Cachau Herreillat redox p 268}, courbe intensité potentiel établie en direct ( notion de potentiel de flade, estimation à l'aide de régressi)
	
	Corrosion du métal plus facile que pour le zinc, car l'oxyde du zinc est imperméable.
	
\subsection{Protection physique : revêtements / anodisation, galvanisation}
	On peut former volontairement un oxyde sur un métal pour le protéger (pylônes, grillage).
	
\section{Corrosion différentielle}
	Définition.
	
\subsection{Approche expérimentale}
\textit{Grésisas p210 ou  Cachau Herreillat redox p167}\\
Cas Fe/Cu et Fe/Zn, dans le premier cas l'électrode de fer est l'anode tandis que dans le second c'est la cathode.\\

Faire une flèche avec les potentiels pour l'expliquer.\\

On trace les courbes intensité-potentiel expliquer pourquoi dans un cas c'est le fer qui est\\ corrodé tandis que dans l'autre c'est le zinc.

\textbf{Manip de la goutte d'Evans}. Plus il y a d'oxygène moins le fer est corrodé, ce qui semble contre intuitif, on va à nouveau l'expliquer à partir des courbes i-e.

\subsection{Protection contre la corrosion / anode sacrificielle}
1ère : établir une liaison avec une source de tension pour placer le métal dans son domaine d'immunité : protection cathodique. Canalisations enterrées car on ne peut y changer l'anode sacrificielle.\\

2e : méthode de l'anode sacrificielle (faire ref au cas du clou en fer entouré de zinc, où c'est alors le zinc qui est corrodé). Méthode très utilisée : coque de navire.

3e : Utiliser un métal plus électropositif que le matériau à protéger pour recouvrir sa surface.

\section*{Questions}
Dans quelle autre filière auriez vous pu proposer cette leçon ?\\
PSI plutôt car leçon appliquée.\\

Formule du ferricyanure ? Nomenclature ?\\
$[Fe(CN)_6]^{3-}$\\

Pourquoi la réduction de l'eau rend le milieu basique ?\\
$2H_2O + 2e^- = H_2 + 2OH^-$\\

Comment se fait le choix de concentration pour les diagrammes potentiel PH ?\\
En général on se place à $10^{-2}$ ou $10^{-3}$ car c'est le domaine de concentrations usuelles en labo. \\

Pourquoi a t-il les oxydes et non les hydroxydes sur le diagramme présenté ?\\
Car ils sont thermodynamiquement plus stables.\\

Quelle différence entre oxyde et hydroxyde ?\\

Que se passe t'il si on met du fer dans de l'eau et que l'on attend ?\\
Le fer va rouiller jusqu'à ce qu'il n'y ait plus du tout de fer.\\

Comment chiffrer la vitesse de corrosion d'un clou dans l'eau (pour la corrosion uniforme) ?\\
Si on a accès à la courbe d'oxydation c'est l'intensité à tension nulle qui nous donne la vitesse : on trace le ln et on regarde le point d'intersection des droites obtenues.\\

A t-on accès expérimentalement à cette courbe d'oxydation ?\\
Non on a accès qu'à la somme.\\

Comment s'appelle le potentiel pour lequel l'intensité est nulle ?\\

Existe t-il un moyen électrochimique de réaliser un dépôt ?\\
Oui comme l'électrozingage par exemple.\\

Pourquoi le zinc recouvrant le fer offre t-il une bonne protection ?\\
Car son oxydation est très lente car l'oxyde est très imperméable.\\

Une autre méthode ?\\
L'anodisation de l'aluminium : on dépose de l'alumine.\\

Quand utilise t-on l'électrozingage plutôt que la galvanisation ?\\
Dépend de la géométrie, dans le cas de l'électrozingage le dépôt est moins homogène (à vérifier)\\

\section*{Remarques}
Domaine de corrosion : l'espèce stable est une espèce en solution. Zone de passivation : l'espèce stable est solide, si elle protège effectivement le métal alors c'est une zone de passivité.\\

regarder histoire statue de la liberté pour illustrer les phénomènes de corrosion.



\end{document}