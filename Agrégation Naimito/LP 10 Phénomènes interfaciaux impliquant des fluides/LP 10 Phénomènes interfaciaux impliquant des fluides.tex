\documentclass[12pt,prb,aps,epsf]{report}
\usepackage[utf8]{inputenc}
\usepackage{amsmath}
\usepackage{amsfonts}
\usepackage{amssymb}
\usepackage{graphicx} 
\usepackage{latexsym} 
\usepackage[toc,page]{appendix}
%\usepackage{listings}
\usepackage{xcolor}
\usepackage{soul}
\usepackage[T1]{fontenc}
\usepackage{amsthm}
\usepackage{mathtools}
\usepackage{setspace}
\usepackage{array,multirow,makecell}
\usepackage{geometry}
\usepackage{textcomp}
\usepackage{float}
\usepackage{cancel}
\usepackage{here}
\usepackage{titlesec}
\usepackage{bbold}

\geometry{hmargin=2cm,vmargin=2cm}

\begin{document}
	
	\title{LP 10 Phénomènes interfaciaux impliquant des fluides}
	\author{Maria}
	
	\maketitle
	
	\tableofcontents
	
	\pagebreak
	
	
\subsubsection{Introduction}
On observe de nouveaux phénomènes interfaciaux. Certains nécessitent, pour être expliqués, de l'introduction d'une grandeur appelée tension superficielle. Images de bulles et effet capillaire (loi de jurin)
\section{Tension superficielle}
\subsection{Définition microscopique}
Une molécule de fluide est en intéraction avec ses voisines, ce qui résulte en une énergie potentielle. Les molécules qui sont à l'interfaces n'ont plus les mêmes intéractions (moitié avec des voisines semblables, et moitié avec l'espèce limitrophe) ce qui est moins favorable énergitiquement. Il va donc y avoir une force qui s'oppose à l'augmentation de la surface de cette interface. Si on note U l'énergie potentielle d'interaction, et a la dimension moyenne d'une molécule, on peut définir $U/2$ l'énergie potentielle d'interaction au niveau de l'interface et ainsi introduire la tension superficielle comme 
\begin{eqnarray}
\gamma \simeq \frac{U}{2a^2}
\end{eqnarray}
\subsection{Description macroscopique}
Expérience : tube au milieu d'un film de savon, on crève le film d'un des deux cotés du tube et on observe un déplacement du tube.\\
On se propose d'expliquer ce phénomène par une modélisation simple, du point de vue mécanique en voyant la tension superficielle comme une force par unités de longueur, puis du point de vue thermodynamique en la voyant comme une énergie de surface.
\section{Interface entre deux fluides : loi de Laplace}
On va étudier le cas de la bulle de savon, du point de vue thermodynamique.
\section{Phénomènes interfaciaux}
\subsection{Mouillage}
Paramètre d'étalement.\\
Angle de mouillage et loi de Young-Dupré.
\subsection{Capillarité}
Introduction au phénomène et principe.\\
On applique la loi de Laplace au niveau du ménisque, que l'on considère comme un morceau de sphère

\section*{Questions}
Pouvez nous dire deux mots sur les ondes capillaires ?\\

La tension superficielle varie t-elle avec la température, et si oui comment ?\\
Elle décroit linéairement avec la température, jusqu'à la température critique, où l'on ne peut plus faire la différence entre liquide et gaz, et où il n'y a donc plus de tension superficielle.\\

Peut elle s'annuler ?\\
Oui, au niveau du point critique et au delà.\\

Quelles sont les applications du mouillage total ?\\
Peinture colle.\\

Phénomènes de capillarité dans la nature ?\\
Sève dans les plantes.\\

Quelle est la pression de la sève au sommet d'un arbre de 100m ? Comment expliquer que cette pression soit négative ?\\
Il faut un pression négative pour faire monter l'eau sur de telles longueurs, l'eau est alors dans un état (très instable) de surfusion (ou surcondensation).

\section*{Remarques}
Mettre la thermodynamique et la MF en pré-requis.\\
Peut être inverser l'ordre de présentation des aspects macroscopiques et microscopiques de la tension superficielle.\\
Renommer la partie 1.1 "Interprétation microscopique".\\
Possibilité de discuter la loi de Laplace avec la caténaire.\\
Le non mouillage peut être utilisé dans la filtration : on fait passer un liquide impur à travers des canaux très fin : si le fluide ne mouille pas les matériaux il passe, tandis que les impuretés ne rentrent pas ou restent bloquées.\\ 

\end{document}