\documentclass[12pt,prb,aps,epsf]{report}
\usepackage[utf8]{inputenc}
\usepackage{amsmath}
\usepackage{amsfonts}
\usepackage{amssymb}
\usepackage{graphicx} 
\usepackage{latexsym} 
\usepackage[toc,page]{appendix}
\usepackage{listings}
\usepackage{xcolor}
\usepackage{soul}
\usepackage[T1]{fontenc}
\usepackage{amsthm}
\usepackage{mathtools}
\usepackage{setspace}
\usepackage{array,multirow,makecell}
\usepackage{geometry}
\usepackage{textcomp}
\usepackage{float}
%\usepackage{siunitx}
\usepackage{cancel}
%\usepackage{tikz}
%\usetikzlibrary{calc, shapes, backgrounds, arrows, decorations.pathmorphing, positioning, fit, petri, tikzmark}
\usepackage{here}
\usepackage{titlesec}
%\usepackage{bm}
\usepackage{bbold}

\geometry{hmargin=2cm,vmargin=2cm}

\begin{document}
	
	\title{LP 39 Aspect ondulatoire de la matière, notion de fonction d'onde}
	\author{Naïmo Davier}
	
	\maketitle
	
	\tableofcontents
	
	\pagebreak
	
	
\subsection{Introduction}
\textbf{Pré-requis :} Optique et électromagnétisme, notion d'onde.\\

On a déjà pu entendre parler d'onde lumineuse et de photon dans des cours et leçons précédentes, ce qui suppose que l'on puisse voir la lumière comme un corpuscule mais aussi comme une onde. Mais qu'en est il pour les autres particules comme les électrons ou les protons par exemple ? Historiquement c'est pour le photon que se sont posées les premières questions qui remontent à loin puisque à l'époque de Newton déjà sa nature faisait débat. Bien que l'électromagnétisme de la fin du 19e ait semblé trancher en faveur d'une nature corpusculaire, certains effets comme le rayonnement de corps noir ou l'effet photoélectrique, expliqué au début du 20e s ont quand à eux soutenu l'autre point de vue. Finalement il est apparu que la lumière n'était pas l'un ou l'autre mais un objet plus complexe pouvant être vu comme une onde ou comme une particule selon la manière dont on l'analyse et dont on interagit avec. La leçon d'aujourd'hui a pour objet d'étendre ce concept aux autres particules.

\section{Aspect ondulatoire de la matière}
\subsection{Comportement ondulatoire des particules}
\textit{1.4.2 et 1.4.3 p20 Le Bellac tome I} : quelles que soient les particules on vérifie qu'elles admettent un comportement ondulatoire : dualité onde corpuscule. Propriétés étonnantes : dans le cas des fentes si on regarder par laquelle des deux fentes passe la particule alors il n'y a plus de figure d'interférence mais seulement la somme des figures de diffraction dues à chacune des fentes séparément, introduit la notion de mesure et de superposition.

\subsection{Relations de de Broglie}
\textit{B-1 chap I p10 Cohen Tannoudji tome I}\\
Donner quelques ordres de grandeur.

\section{Formalisme quantique}
\subsection{Fonction d'onde}
\textit{B-2 chap I p11 Cohen Tannoudji tome I}
\subsection{Équation de Schrödinger}
\textit{B-2 chap I p11 Cohen Tannoudji tome I}
\subsection{Particule libre}
\textit{C chap I p14 Cohen Tannoudji tome I}
\subsection{Inégalité de Heisenberg}
\textit{C-3 chap I p20 Cohen Tannoudji tome I}\\
Donner quelques ordres de grandeur pour comparer le cas d'une particule avec celui d'une poussière : \textit{Complément $B_I$ chap I p41 Cohen Tannoudji tome I}, en déduire qu'on ne peut observer cette dualité onde corpuscule à l'échelle macroscopique.\\
Possibilité de parler de l'hydrogène : \textit{8.1.4 p250 Le Bellac tome I}
\subsection{Particule liée}
\textit{Paragraphe 8.3.3 p262 Le Bellac tome I}

\section{Conclusion}
On a pu aujourd'hui une manière nouvelle de percevoir la matière aux échelles microscopique au travers de la dualité onde corpuscule. 

\chapter{Plan Hugo}	
\section{Aspect ondulatoire de la matière}

\subsection{Relation de de Broglie}

Énoncé des principes de Fermat et de Maupertuis, cela amène $p =\alpha k$, où on va définir la constante à partir de la constante de Planck tel que $\lambda = \frac{h}{p}$.\\
On donne quelques ordre de grandeur de longueur d'onde de De Broglie pour expliquer pourquoi on ne perçoit pas la nature ondulatoire de la matière à l'échelle macroscopique.\\
On établit le $L= n \hbar$ du modèle de bohr en partant de $2\pi r = n\lambda$.

\subsection{Expérience de Davisson et Germer}
\subsection{Diffraction d'électrons par le graphite}
On réalise l'expérience en direct, et on observe une figure d'interférence composée d'anneaux.\\
On l'explique à partir d'un schéma, à partir duquel on établit la relation de Bragg.\\
On montre ensuite un schéma expliquant la manière dont les électrons sont accélérés. On en déduit leur énergie et ainsi leur longueur d'onde de De Broglie. Cela permet de lier l'interfrange à la distance entre les couches de graphite. On mesure donc l'interfrange pour la calculer. En comparant le résultat obtenu aux valeurs tabulées on peut ainis vérifier le formalisme de de Broglie.

\section{Notion de fonction d'onde}
\subsection{Expérience des électrons}

Interférences entre des électrons à travers des trous d'Young. Pas dues aux intéractions électron-électron puisque la figure d'interférence peut être obtenue en envoyant les lectrons un par un à travers les fentes d'Young. Modification de la figure lorsque l'on ferme l'une des deux fentes. On constate que la figure avec deux fentes n'est pas la somme des figures d'interférence produites par les deux fentes séparément.\\
On en déduit que notre interprétation classique n'est pas adapté aux cas quantiques, ainsi que notre formalisme.

\subsection{Fonction d'onde}
On va donc introduire un formalisme permettant de rendre compte de la réalité quantique : la fonction d'onde, qui est intrinsèquement probabiliste.\\
La description complète de l'état quantique d'une particule de masse m dans l'espace et à l'instant t se fait au moyen d'une fonction d'onde complexe $\psi (\vec{r}, t)$.\\
Interprétation de Born : probabilité de trouver la particule à la position $\vec{r}$ à $d^3r$ près : $|\psi (\vec{r}, t)|^2d^3r$.

\subsection{Équation de Schrödinger}
Cette équation est déterministe : à partir de la donnée de la fonction d'onde à un instant initial $t_0$, et du hamiltonien à chaque instant on est capable de connaitre la valeur de la fonction d'onde à tout instant $t>t_0$.\\
Elle est aussi réversibleet linéaire.\\
En considérant une onde plane et en l'injectant dans l'équation de Schrodinger on est en mesure d'identifier les parties cinétique et potentielles du hamiltonien.

\section{Applications et conséquences}
\subsection{Particule dans un puits infini}
\subsection{Paquet d'onde}
\subsection{Inégalités d'Heisenberg}


\section*{Questions}
Peut on avoir le concept d'inégalité d'Heisenberg sans transformé de fourrier ou est il intrinsèquement lié à ce formalisme ?\\
Si on prend A et B tel que $[A,B]\neq 0$, alors pour tout $\psi$ on a 
\begin{eqnarray}
\Delta A \Delta B \geq \frac{|\langle[A,B]\rangle|}{2}
\end{eqnarray}

Lors d'un déplacement de le l'onde, le paquet d'onde ne fait il qu'une translation ?\\
Non il existe une notion d'étalement du paquet d'onde, qui apparait au deuxième ordre. (Il évoquer l'étalement du paquet gaussien).\\

Quelle est l'analogie optique que l'on peut faire avec le puits infini ?\\
On peut parler de cavité optique et de guide d'onde.\\

Pourquoi les premières observations ont elles été faites avec des électrons ?\\
Parce que les électrons sont bien maitrisés à cette époque et ont une très faible masse.\\

Pour le canon à l'électron, l'énergie cinétique est elle toujours $p^2/2m$ ?\\
Non il faut prendre en compte les effets relativistes à priori.\\

A t'on déjà diffracté ou fait interférer des atomes ?\\

Un gaz entier peut il avoir un caractère ondulatoire ?\\
Oui, dans le cas d'un condensat de Bose-Einstein.\\

Tout est il réversible en mécanique quantique ?\\
Non, l'irréversibilité est introduite par la mesure et la réduction du paquet d'onde qu'elle entraine.\\

Peut on faire interférer des particules différentes ?\\
Oui, c'est l'effet Ou-Mandel où l'on envoie deux photons sur les faces d'une lame semi réfléchissante et ils se retrouvent toujours tous les deux du même coté et jamais chacun un de chaque côté.

\section*{Remarques}
Lorsqu'on évoque l'atome d'hydrogène et le modèle de Bohr, il faut dire que l'on considère l'électron dans le puits de potentiel formé par le proton : on raccroche avec la quantification de l'énergie dans le puits de potentiel /cavité.\\
Ne pas faire de mesures en direct pour la manip, car prend trop de temps. Vérifier le caractère relativiste ou non des électrons.\\
Il faut donner les dates pour chaque grand principe ou grande équation énoncée.\\
Pour le corps de la leçon, il faut plus de structure et de motivations : on doit justifier chaque partie.\\
Possibilité de faire l'analogie avec l'électricité : $E\Leftrightarrow\psi$ pour introduire la fonction d'onde.\\
Introduire l'équivalence mécanique ondulatoire via les quadrivecteurs pour avoir une équivalence totale.\\
Attention, lors de l'évocation de la quantification du moment cinétique de l'atome d'hydrogène, et des ondes propagatives précédentes : ce ne sont pas les mêmes types d'ondes qui interviennent.\\
Regarder le Dalibard pour la discussion dur les ordres de grandeur.\\
Il faut préciser que l'expérience de Davisson et Germer suit et est totalement inspirée de celle similaire sur les rayons X.	
	
\end{document}