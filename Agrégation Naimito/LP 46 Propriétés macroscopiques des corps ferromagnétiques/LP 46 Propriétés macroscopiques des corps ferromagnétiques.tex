\documentclass[12pt,prb,aps,epsf]{article}
\usepackage[utf8]{inputenc}
\usepackage{amsmath}
\usepackage{amsfonts}
\usepackage{amssymb}
\usepackage{graphicx} 
\usepackage{latexsym} 
\usepackage[toc,page]{appendix}
\usepackage{listings}
\usepackage{xcolor}
\usepackage{soul}
\usepackage[T1]{fontenc}
\usepackage{amsthm}
\usepackage{mathtools}
\usepackage{setspace}
\usepackage{array,multirow,makecell}
\usepackage{geometry}
\usepackage{textcomp}
\usepackage{float}
%\usepackage{siunitx}
\usepackage{cancel}
%\usepackage{tikz}
%\usetikzlibrary{calc, shapes, backgrounds, arrows, decorations.pathmorphing, positioning, fit, petri, tikzmark}
\usepackage{here}
\usepackage{titlesec}
%\usepackage{bm}
\usepackage{bbold}

\geometry{hmargin=2cm,vmargin=2cm}

\begin{document}
	
	\title{LP 46 Propriétés macroscopiques des corps ferromagnétiques}
	\author{Naïmo Davier}
	\date{Agrégation 2019}
	
	\maketitle
	
	\tableofcontents
	
	\pagebreak
		
\subsection{Introduction}
Propriétés des ferros. Existence de nombreuses applications.

\section{Rappels}
\textbf{Electromagnétisme tome 4 : milieux diélectriques et milieux aimantés} de \textit{Bertin, Faroux et Renault} p 124.\\

Il faut être rapide.
\subsection{Vecteur aimantation}
Définition :
\begin{eqnarray}
\vec{M} = \frac{\sum\vec{m}_i}{dV},
\end{eqnarray}
unités : A.M$^{-1}$. On peut définir une densité volumique de courant d'aimantation : 
\begin{eqnarray}
\vec{j}_m = \vec{rot}\times \vec{M},
\end{eqnarray} 
Expérience bobine et aimant : en faisant passer un courant dans la bobine on génère un champ magnétique qui perturbe la boussole.\\
On va maintenant chercher à établir le lien entre aimantation et champ magnétique.

\subsection{Relation aimantation-champ $\vec{B}$}
\begin{eqnarray}
rotB = \mu_0(j_m+j_a) = \mu_0(rotM+j_a)
\end{eqnarray}
On définit alors 
\begin{eqnarray}
H = \frac{B}{\mu_0}-M
\end{eqnarray}
tel que 
\begin{eqnarray}
\vec{rot}\vec{H} = \mu_0\vec{j}_a
\end{eqnarray}
 On a donc une relation cyclique entre champ et aimantation : l'un agit sur l'autre et inversement.
 
 \subsection{Susceptibilité du champ magnétique}
def : $M=\chi_m H$\\
$H = B/\mu_0 - M = B/\mu_0 - \chi_m H$ $\Rightarrow$ $B = \mu_0(1+\chi_m)H = \mu_0\mu_r H$ et de même $M = (\frac{1}{\mu_0}-\frac{1}{\mu_0\mu_r})B$\\
Ordres de grandeurs : ferro cas du fer $\mu_r \simeq 5.10^5$, mimetal $\mu_r \simeq 3.10^4$ 

\section{Etude expérimentale d'un ferromagnétique}
\textbf{Electromagnétisme tome 4 : milieux diélectriques et milieux aimantés} de \textit{Bertin, Faroux et Renault} p 136.

\subsection{Modèle du tore magnétique}
Présenter le modèle, conclure en explicitant l'aspect pratique du dispositif : on a accès à H et B. On va l'appliquer en direct.

\subsection{Réalisation pratique}
\textbf{Manip : Courbe de première aimantation + cycle d'hystérésis}\\
Présentée dans le \textbf{Quaranta} \textit{tome IV Électricité et applications} à la partie transformateurs p582. Voir montage 16 milieux magnétiques.

\subsection{Perte fer}
\textbf{Electromagnétisme tome 4 : milieux diélectriques et milieux aimantés} de \textit{Bertin, Faroux et Renault} p 181 et 240.\\
\textbf{Magnétisme I-Fondements} de \textit{E du Trémolet} p226.\\

Notion de coercitivité. Calcul de la puissance
\begin{eqnarray}
P = RI^2 + V\int HdB
\end{eqnarray}
avec $V$ le volume du ferro.\\

En déduire si le ferro étudié est dur ou doux après avoir défini les deux catégories.

\subsection{Interprétation}
\textbf{Magnétisme I-Fondements} de \textit{E du Trémolet} et \textbf{Electromagnétisme tome 4 : milieux diélectriques et milieux aimantés} de \textit{Bertin, Faroux et Renault} chap 8.\\

Domaines de Weiss, dont l'existence a été vérifiée au microscope, taille des domaines, notion de parois de Block.\\
Interprétation de la courbe d'hystérésis avec des schémas illustrant les domaines de Weiss.

\section{Applications}
\subsection{Disque dur magnétique}
\textbf{Magnétisme I-Fondements} de \textit{E du Trémolet} pour la théorie,
Regarder aussi le \textbf{tome II-Applications}.\\
Pour la pratique regarder sur internet :\\
http://tpe.kyvandoan.free.fr/?page\_id=24\\
https://www.futura-sciences.com/tech/dossiers/informatique-stockage-donnees-informatiques-105/page/3/\\

Ou encore regarder le problème de 2019 de l'agrégation externe.

\subsection{Magnétorésistance}
\textbf{Magnétisme I-Fondements} de \textit{E du Trémolet} p 442.\\

On peut aussi parler du transformateur, ou du paléomagnétisme (Voir \textbf{Magnétisme II-Applications} de \textit{E du Trémolet} (attention c'est le tome 2 cette fois))

\section{Conclusion}
Énorme intérêt pratique de ces matériaux.

\section*{Questions}
Au vu du titre de la leçon, pourquoi commencer par parler des relations constitutives ? \\

Que se passe t-il si on met un diamagnétique ou un paramagnétique pour l'expérience de la boussole, qu'aurait on observé ?\\
Rien.\\

Pourquoi utiliser le modèle du tore magnétique ?\\
Parce qu'il permet de contrôler H. \\

Pourquoi le ferro canalise t-il les lignes de champs dans la manip ?\\
Condition de passage du champ : B est nul à l'extérieur au départ, et $B_{\perp}$ étant continu, le champ va s'aligner selon la direction du barreau.\\

Comment faut il feuilleter les matériaux pour la manip ?\\
Dans la longueur pour éviter les courants de Foucault qui se font dans les plans orthogonaux au champ qui est guidé.

\section*{Remarques}
Il faut enlever la première partie. Laisser la def de la susceptibilité en intro, et introduire en montrant qu'il existe des ferro durs et doux.\\
Montrer expérimentalement en intro la différence entre ferro et ferro dur en collant un trombone.\\
Il faut parler de l'existence de ferro fluides, fait de colloïdes.\\
C'est bien de faire la courbe de première aimantation, il faut donc faire les explications pour le tore. Il faut alors bien montrer que l'asymptote est oblique dans le cas $B(H)$ (contrairement au cas $M(B)$ où elle est horizontale).\\
Mettre la section interprétation plus tôt : permet d'introduire la notion de perte fer : une fois les spins alignés, il va falloir dépenser de l'énergie pour les retourner.\\
Regarder la magnétorésistance.

\end{document}