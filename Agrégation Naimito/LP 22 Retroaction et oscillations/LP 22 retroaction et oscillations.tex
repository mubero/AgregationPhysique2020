\documentclass[12pt,prb,aps,epsf]{report}
\usepackage[utf8]{inputenc}
\usepackage{amsmath}
\usepackage{amsfonts}
\usepackage{amssymb}
\usepackage{graphicx} 
\usepackage{latexsym} 
\usepackage[toc,page]{appendix}
\usepackage{listings}
\usepackage{xcolor}
\usepackage{soul}
\usepackage[T1]{fontenc}
\usepackage{amsthm}
\usepackage{mathtools}
\usepackage{setspace}
\usepackage{array,multirow,makecell}
\usepackage{geometry}
\usepackage{textcomp}
\usepackage{float}
\usepackage{cancel}
\usepackage{here}
\usepackage{titlesec}
\usepackage{bbold}

\geometry{hmargin=2cm,vmargin=2cm}

\begin{document}
	
	\title{LP 22 Rétroaction et oscillations}
	\author{Matthieu}
	
	\maketitle
	
	\tableofcontents
	
	\pagebreak
	
	
\section{Introduction}
Notion d'interaction, notion de système.
Interaction entre ce système et son environnement.\\
Notion de système bouclé : qui contiens une ou des boucles de rétroaction.\\
Définition d'un système linéaire.
\section{Modélisation des systèmes bouclés - Asservissement}
\subsection{Structure d'un système bouclé}
Ex adaptation vitesse par un conducteur dans une voiture.\\
Def asservissement.\\
Schéma + explications d'un régulateur de vitesse.\\
Correspond à une chaîne directe + une chaîne de retour
\subsection{Modélisation d'un système bouclé}
Système linéaire :
\begin{eqnarray}
a_k\frac{d^k S}{dt^k} + ... + a_0S=b_l\frac{d^l e}{dt^l} + ... + b_0e
\end{eqnarray}
Resolution de ce type d'équations en utilisant la transformée de Laplace
\begin{eqnarray}
E(p)=TL(e(t))=\int_0^{+\infty}e^{pt}e(t)dt\;avec \; p\in\mathcal{C}\\
TL\frac{d^kS}{dt^k}=p^kS(p)
\end{eqnarray}
Application à une eq linéaire.
\begin{eqnarray}
H(p)=\frac{S(p)}{E(p)}
\end{eqnarray}
Fonction de transfert en Boucle ouverte ...
\subsection{Application au régulateur}

\section{Stabilité des systèmes bouclés, Oscillateurs}
\subsection{Étude de la stabilité}
Cas général ... stabilité dépend de la partie réelle des pôles de S(p) la TL de S.
\paragraph{Exemple du régulateur de vitesse}
\subsection{Oscillateurs quasi-sinusoïdaux}
Cas de l'oscillateur à pont de Wien. Shéma + principe + caractérisation.

\section{Questions}
Qu'est ce qu'une contre réaction (différence avec la rétroaction)?\\

Peut on modéliser un AOP en montage amplificateur non-inverseur comme un système bouclé ?\\

Avantages des systèmes bouclés ?\\

Pouvez vous modéliser analytiquement et proprement l'impact d'un bruit sur un système bouclé ?
\end{document}