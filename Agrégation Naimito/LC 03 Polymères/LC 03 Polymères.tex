\documentclass[12pt,prb,aps,epsf]{report}
\usepackage[utf8]{inputenc}
\usepackage{amsmath}
\usepackage{amsfonts}
\usepackage{amssymb}
\usepackage{graphicx} 
\usepackage{latexsym} 
\usepackage[toc,page]{appendix}
%\usepackage{listings}
\usepackage{xcolor}
\usepackage{soul}
\usepackage[T1]{fontenc}
\usepackage{amsthm}
\usepackage{mathtools}
\usepackage{setspace}
\usepackage{array,multirow,makecell}
\usepackage{geometry}
\usepackage{textcomp}
\usepackage{float}
\usepackage{cancel}
\usepackage{here}
\usepackage{titlesec}
\usepackage{bbold}

\geometry{hmargin=2cm,vmargin=2cm}

\begin{document}
	
	\title{LC 03 Polymères}
	\author{Etienne}
	
	\maketitle
	
	\tableofcontents
	
	\pagebreak
	
\subsection{Pré-requis}
Niveau 1ère STL-STI2D\\
programme de seconde\\
Notion de synthèse.

\subsection{Introduction}
Les polymères se sont imposés comme indispensables dans de ombreux domaines. Plastiques, téflon, kevlar, PVC etc...\\
Grande plasticité, facilité de synthèse, mais problèmes environnementaux.\\
Citons l'exemple du nylon : synthèse en direct, voir \textit{Le maréchal} \textbf{Tome 2 Chimie organique et minérale} p 119.

\section{Du monomère au polymère}
\subsection{Quelques définitions}
\paragraph{Polymère} Ensemble de macromolécules elles mêmes constituées d'un grand nombre de répétitions d'un motif :
\begin{eqnarray}
&H&\hspace{0.45cm}H \nonumber \\
&|&\hspace{0.5cm}| \nonumber \\
-&C&-\;C-\\
&|&\hspace{0.5cm}| \nonumber \\
&H& \hspace{0.45cm}Cl \nonumber 
\end{eqnarray}
\paragraph{Monomère} $CH_2=CHCl$\\
\paragraph{Degré de polymérisation}
\begin{eqnarray}
n = \frac{M(polymere)}{M(monomere)}
\end{eqnarray}
\paragraph{Notion de fonctionnalité}

\subsection{Retour sur la synthèse : polymérisation par polycondensation}
Illustration des étapes avec des atomes et liaisons colorées en plastique + schémas réactionnels.\\
Évocation des polyesters.

\subsection{Polymérisation par polyaddition}
\textit{Le maréchal} \textbf{Tome 2 Chimie organique et minérale} p107 pour les schémas et explication, mais suivre le protocole joint dans le dossier.\\

Schéma de la manip de la synthèse du polystyrène, et précipitation + filtrage
 et calcul du rendement en direct.\\
 Explication des réactions en jeu.\\
 
\section{Du micro au macro : propriétés}
\subsection{Liaisons intermoléculaires : propriétés thermiques}
Cas général.\\
Thermoplastiques.\\
Thermodurcissables.
\subsection{Structure : propriétés mécaniques}
Graphe représentant $\sigma =  \frac{f}{S_0}$ en fonction de la déformation $\varepsilon = \frac{\Delta \rho}{\rho_0}$, pour différents types de matériaux. Explications.
\section{Conclusion}
Résumé des points principaux.

\section*{Questions}
Quelle est la nuance entre polymère naturel, synthétique et artificiel ? Quelques exemples ? \\
Polymère naturel se trouve dans la nature : ex : latex. polymère artificiel = polymère naturel modifié : ex : explosifs, balles de ping pong (celluloïdes). Polymère synthétique : entièrement créé par l'homme.\\

Intérêt des PVC ? pourquoi les polymères sont ils toujours associés à des plastifiants ?\\
Très malléable : on peut en faire ce que l'on veut.\\

A quoi sert la soude lors de la synthèse du nylon ?\\
Pour neutraliser l'acide chlorhydrique libéré, et ainsi ne pas protoner l'amine, qui ne pourra pas attaquer alors (car elle ne sera plus nucléophile)\\

Qu'est ce que le degré de polymérisation ?\\
C'est le nombre de monomères/motifs contenus dans le polymères.\\

Un polymère est il un corps pur ?\\
Non c'est un mélange car il y a plusieurs longueurs de chaines.\\

Le styrène est il soluble ? Le polystyrène l'est il ?\\

Pourquoi pour le rendement avez vous fait le rapport des masses ? Pouvez vous écrire l'équation de réaction ?\\
Tous les monomères vont êtres inclus (pendant la réaction) dans des chaînes ou polymères : le rendement est donc ici le rapport des masses (puisque l'on néglige la masse molaire des initiateurs de radicaux).\\

Comment la chaîne commence t-elle à se former ? Pouvez vous dessiner ce qu'il se passe ?\\

Comment deux "morceaux" de chaînes se lient ils (avec chacun un initiateur de radicaux) ? Comment une chaîne peut elle se scinder ?\\

Comment pourrait on déterminer la masse molaire d'un polymère ?\\
Chromatographie par exclusion stérique. Ou alors on regarde comment la solution de polymères s'écoule, sachant que l'on peut établir un lien entre viscosité et masse molaire.\\

En terme de stéréochimie : y a t'il différentes configurations pour les polymères ? Quels sont les différences au niveau de la stabilité et des propriétés chimiques et physiques ?

\section*{Remarques}
Attention avec l'utilisation de modèles, les éléments intermédiaires ne sont pas stables/fermés, il faut bien montrer que ce sont des étapes et que les molécules intermédiaires formées vont tout de suite réagir à nouveau.\\
Comme on est au niveau lycée on peut ne pas trop rentrer dans les détails, et simplement dire que les doubles liaisons s'ouvrent pour lier les styrènes en formant des liaisons simples.\\

Sources : \textit{Le maréchal} \textbf{Tome 2 chimie organique et minérale}\\
\textbf{L'indispensable en polymères} de \textit{Christophe Chassanieux}


\end{document}