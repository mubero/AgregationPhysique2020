\documentclass[12pt,prb,aps,epsf]{article}
\usepackage[utf8]{inputenc}
\usepackage{amsmath}
\usepackage{amsfonts}
\usepackage{amssymb}
\usepackage{graphicx} 
\usepackage{latexsym} 
\usepackage[toc,page]{appendix}
\usepackage{listings}
\usepackage{xcolor}
\usepackage{soul}
\usepackage[T1]{fontenc}
\usepackage{amsthm}
\usepackage{mathtools}
\usepackage{setspace}
\usepackage{array,multirow,makecell}
\usepackage{geometry}
\usepackage{textcomp}
\usepackage{float}
%\usepackage{siunitx}
\usepackage{cancel}
%\usepackage{tikz}
%\usetikzlibrary{calc, shapes, backgrounds, arrows, decorations.pathmorphing, positioning, fit, petri, tikzmark}
\usepackage{here}
\usepackage{titlesec}
%\usepackage{bm}
\usepackage{bbold}

\geometry{hmargin=2cm,vmargin=2cm}

\begin{document}
	
	\title{LC 12 Molécules de la santé}
	\author{Maxime}
	
	\maketitle
	
	\tableofcontents
	
	\pagebreak
	
\subsection{Pré-requis}
Réactions acido basiques et redOx, isoméries, formules topologiques.\\
Niveau lycée (1ère STL).

\subsection{Intro}
Historiquement les molécules thérapeutiques sont utilisées depuis l'antiquité au travers de plantes médicinales. Aujourd'hui on référence et créé ces molécules artificiellement.

\section{Les médicaments}
\subsection{Quelques définitions}
Médicament : toute substance qui possède des pouvoir curatifs pour les maladies. Il peut être naturel ou synthétique.\\
La molécule qui est responsable de l'effet bénéfique est appelée principe actif : "molécules ayant des propriétés thérapeutiques".\\
Ces molécules sont synthétisées par des labo qui déposent alors des brevets : "princeps", et qui privatise alors la molécule pour une dizaine d'année. Après cette durée on verra apparaître des génériques : médicament basé sur la même molécule, mais fait par un autre labo.\\

On ajoute des excipients aux médicaments pour modifier le goût, la forme et texture, la vitesse d'action sur le corps...ect.\\
Exemple : effervescence
\begin{eqnarray}
HCO_3^- + R-COOH \rightarrow CO_{2(g)} + R_COO^- + H_2O
\end{eqnarray}

Formulation : liste des composés avec doses (forme générique).\\

Il faut en général voir que selon la dose le principe actif peut être bénéfique ou agir comme un poison.

\subsection{Géométrie des molécules}
En général deux isomères n'ont pas les mêmes propriétés. Exemple :
\begin{eqnarray}
C_4H_{10}O \;:\; \mathrm{peut\;etre\;du\;}\textbf{diethylether}\; \mathrm{ou \;du\;}\textbf{butan-1-ol}
\end{eqnarray}
Autre exemple : thalidomide : médicament contre la nausée.\\

Le problème c'est que même si l'on ingère qu'un isomère, le mélange va se racémiser dans le corps.

\subsection{Aspirine}
Utilisée pour les maux de têtes notamment. Le principe actif est l'acide acétylsalicilique. \\
\textit{Manip : Synthèse de l'aspirine, }\textbf{Le maréchal tome 2 p151}.\\
Filtrage en direct.\\
Calcul du rendement à partir du produit recristallisé fait en préparation.\\
Caractérisation banc Koffler.

\section{Antiseptiques et désinfectants}
Tuent les germes pouvant causer des maladies.\\
Antiseptique : s'applique sur de la matière vivante (élimine des germes sur une plaie par exemple).\\
Désinfectant : s'applique sur de la matière inerte (pour éliminer des germes sur des objets comme de l'eau de javel par exemple).\\

Exemples : l'eau oxygénée (mercurochrome), diiode, permanganate, ions hypochlorite (eau de javel).\\
Point commun : ce sont tous des oxydants. (donner les couples).\\

\textit{Dosage du diiode contenue dans la bétadine}, \textbf{Cachau Héreillat : RedOx.}\\

\textbf{Conclusion :} Ouverture sur les médicaments frauduleux, et mentionner que cela fait partie des domaines où de gros efforts sont à faire en terme de chimie verte.\\
On peut aussi ouvrir sur la construction de nano-structures pour cibler l'effet des molécules.

\section*{Questions}
Lors de la synthèse de l'aspirine, comment choisis t-on les proportions initiales ?\\
L'acide salicylique doit être le réactif limitant pour que le produit final soit pur.\\

Comment aurait on pu vérifier la pureté du produit ?\\
Chromato sur couche mince : comparaison avec un comprimé industriel.\\

La réaction est elle totale en théorie ?\\
Oui.\\

Pourquoi un rendement si faible alors ?\\
Parce qu'on perd beaucoup en recristallisant.\\

L'aspirine est elle soluble dans l'eau ? Connaissez vous un ordre grandeur de la solubilité dans l'eau ?\\
qq g/L.\\

Que devient l'acide acétylsalicylique dans un comprimé effervescent lorsqu'on met le comprimé dans l'eau ?\\
Équation (1) en remplaçant l'acide par l'acide acétylsalicylique. Il se retrouve donc sous forme basique, qui est beaucoup plus soluble.\\

Que se passe t-il alors dans l'estomac ?\\
Il y a recristallisation puisque le Ph de l'estomac est très acide.\\

Nom de $S_2O_4^{2-}$ ?\\
tetrathionate\\

\section*{Remarques}
C'est bien de préciser qu'en pratique on part du phénol avant d'arriver à l'acide salicylique.\\

Insister que l'aspirine obtenue doit être pure. Mentionner la CCM pour le vérifier.\\

Introduire le terme posologie.\\

Commenter le fait que dans le cas de l'aspirine il n'y a pas de problème de chiralité.\\



	
\end{document}