\documentclass[12pt,prb,aps,epsf]{article}
\usepackage[utf8]{inputenc}
\usepackage{amsmath}
\usepackage{amsfonts}
\usepackage{amssymb}
\usepackage{graphicx} 
\usepackage{latexsym} 
\usepackage[toc,page]{appendix}
\usepackage{listings}
\usepackage{xcolor}
\usepackage{soul}
\usepackage[T1]{fontenc}
\usepackage{amsthm}
\usepackage{mathtools}
\usepackage{setspace}
\usepackage{array,multirow,makecell}
\usepackage{geometry}
\usepackage{textcomp}
\usepackage{float}
%\usepackage{siunitx}
\usepackage{cancel}
%\usepackage{tikz}
%\usetikzlibrary{calc, shapes, backgrounds, arrows, decorations.pathmorphing, positioning, fit, petri, tikzmark}
\usepackage{here}
\usepackage{titlesec}
%\usepackage{bm}
\usepackage{bbold}

\geometry{hmargin=2cm,vmargin=2cm}

\begin{document}
	
	\title{LP 49 Oscillateurs, portraits de phase et non linéarités}
	\author{Matthieu}
	\date{Agrégation 2019}
	
	\maketitle
	
	\tableofcontents
	
	\pagebreak
	
\subsection{Introduction}
	Systèmes linéaires usuels en physique mais rares dans la pratique.\\
	Notion de linéarisation, car systèmes linéaires solubles.\\
	Résolution numérique pour les systèmes non linéaires.
	
\section{Pendule pesant}
\subsection{Aspects énergétique}
\begin{eqnarray}
\varepsilon_k = \frac{1}{2}ml^2\dot{\theta}^2 + \frac{1}{2}I\dot{\theta}^2\\
\varepsilon_p = \left(m+\frac{M}{2}\right)gl\cos\theta
\end{eqnarray}
	On peut tracer l'énergie mécanique, et ainsi regarder le comportement du système selon la valeur de cette dernière.
	
On regarde l'équation $\frac{dE}{dt}=0$ pour obtenir l'équation du mouvement : 
\begin{eqnarray}
\ddot{\theta} = \omega_0^2\sin\theta
\end{eqnarray}

On peut regarder cette équation à l'aide d'un portrait de phase $(\theta,\dot{\theta}/\omega_0)$.

\subsection{Étude dans l'espace des phases}
	
	On peut tracer numériquement le portrait de phase du pendule en prenant différentes conditions initiales.\\
	
	On obtient 3 types de figures : des cercles (ou presque), des "yeux", et des "vagues".\\
	Le cercle correspond aux petites oscillations pour lesquelles on a $\theta$ de la forme $\theta = \cos(\omega_0 t)$. Analogie avec tous les développements harmoniques des potentiels possédant des minimum.\\
	
\subsection{Du linéaire au non linéaire}
	Cas du développement du potentiel à l'ordre 4 où on a alors une équation non linéaire.\\
	Formule de Borda :
	\begin{eqnarray}
	T = T_0\left(1 + \frac{\theta_{max}^2}{18}\right)
	\end{eqnarray}
	
	Expérience avec pendule pesant type Montage non linéaire pour les docteurs, où l'on trace $T(\theta_{max})$, que l'on compare avec le formule de Borda.\\
	
	On se propose maintenant de redémontrer la formule de Borda :\\
	On développe le sinus à l'ordre 3 en $\theta$. On va ensuite chercher $\theta$ sous la forme $\theta = \sin\omega t$, en utilisant alors que 
	\begin{eqnarray}
	4\sin^3\omega t = 3\sin \omega t - \sin 3\omega t
	\end{eqnarray}
	On cherche alors $\theta$ sous la forme
	\begin{eqnarray}
	\theta = \theta_m(\sin\omega t + \varepsilon\sin 3\omega t)\hspace{0.5cm}\varepsilon, \theta_m \ll 1
	\end{eqnarray}
	On en déduit la formule de Borda.
	
\section{Oscillateur à double puits}
\subsection{Cas sans dissipation, bifurcation}
Oscillateur de Duffing : passage du puits simple et donc une position d'équilibre à double puits et donc 3 positions d'équilibre. On appelle ça une bifurcation, donner la définition générale.\\

Illustration numérique.

\subsection{Introduction de la dissipation}

\section{Comportement chaotique}
Définition.\\
Illustration numérique du chaos.\\
Horizon de prédictibilité. \\
Exemple du système solaire.

\section*{Questions}
Existe t-il un cas réel où on ait un potentiel de Duffing avec $\alpha$ positif et $\beta$ négatif ?\\
Non, c'est nécessairement une approximation locale car on ne peut avoir $E\rightarrow \infty$.\\

Comment définit-on le fait qu'un système soit chaotique ?\\
Sensibilité aux conditions initiales : coefficients de Lyapunov positifs, c'est à dire que la différence entre deux solution croît exponentiellement avec le temps.\\

L'une de ces conditions seule suffit elle à définir le système comme chaotique ?\\
Non.\\

Quelle est la différence entre instable et chaotique ? Des exemples ?\\
L'instabilité peut apparaître avec des systèmes linéaires.\\

Pourquoi considérer un pendule pesant et non un pendule simple ?\\
Pour la manipulation qui utilise un pendule pesant.\\

Quelle est la condition pour que l'énergie se conserve ?\\
Il faut que les forces qui travaillent soient conservatives.\\

Dans le cas du pendule, quelles sont les actions mécaniques, et les quelles travaillent ?\\
Les actions au niveau des liaisons ne travaillent pas, le poids travaille.\\

Comment définit on une action mécanique ?\\

Comment détermine t-on le nombre de degrés de liberté d'un système ?\\
Nombre de paramètre dont on a besoin pour décrire le mouvement, c'est la variance.\\

Pouvez vous expliquer votre programme (partie numérique) ?\\

\section*{Remarques}
Il faut montrer les sens de parcours sur les diagrammes de phase.\\

Ce qui a été correspond à toujours traiter un potentiel avec un terme quadratique et un terme d'ordre 4.\\

On aurait pu le faire avec un terme d'ordre 3 plutôt que d'ordre 4.\\

Possibilité de regarder le Queré de physique des matériaux (livre jaune) pour une interprétation statistique de la position moyenne dans un potentiel. On peut aussi regarder le Landeau si on a du courage...\\

Possibilité de regarder l'effet du forçage sur la résonance dans le cas non linéaire, regarder le Landeau, l'effet Kerr optique dans le Garing ou le livre de Paul Maneville.\\


	
\end{document}