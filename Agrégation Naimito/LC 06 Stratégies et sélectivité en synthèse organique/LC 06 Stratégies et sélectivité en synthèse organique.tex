\documentclass[12pt,prb,aps,epsf]{article}
\usepackage[utf8]{inputenc}
\usepackage{amsmath}
\usepackage{amsfonts}
\usepackage{amssymb}
\usepackage{graphicx} 
\usepackage{latexsym} 
\usepackage[toc,page]{appendix}
\usepackage{listings}
\usepackage{xcolor}
\usepackage{soul}
\usepackage[T1]{fontenc}
\usepackage{amsthm}
\usepackage{mathtools}
\usepackage{setspace}
\usepackage{array,multirow,makecell}
\usepackage{geometry}
\usepackage{textcomp}
\usepackage{float}
%\usepackage{siunitx}
\usepackage{cancel}
%\usepackage{tikz}
%\usetikzlibrary{calc, shapes, backgrounds, arrows, decorations.pathmorphing, positioning, fit, petri, tikzmark}
\usepackage{here}
\usepackage{titlesec}
%\usepackage{bm}
\usepackage{bbold}
\geometry{hmargin=2cm,vmargin=2cm}

\begin{document}
	
	\title{LC 06 Stratégies et sélectivité en synthèse organique}
		\author{Maxime}
		\date{Agrégation 2019}
		
	\maketitle
	
	\tableofcontents
	
	\pagebreak

\subsection*{Introduction}
Terminale S, réactions acides bases etc...\\

Les protocoles de synthèse que l'on a déjà pu rencontrer ont été construit selon certaines règles et contraintes que nous allons étudier aujourd'hui.

\section{Stratégie}
\subsection{Étapes d'une synthèse}
On va traiter la synthèse de l'aspirine.\\ 
Description des différentes phases de la synthèse : préparation, séparation, identification ... etc

\subsection{Rendement}
\begin{eqnarray}
\eta = \frac{n(produit)}{n_{max}(produit)}
\end{eqnarray}
Exemple de la synthèse de l'éthanoate de benzyle.\\ 
On donne l'équation de la synthèse, les masses des réactifs, on identifie le réactif limitant.\\
On utilise la masse de produit obtenu en préparation pour faire un calcul de rendement, on trouve 67\%.\\

L'objectif va être ici de maximiser ce rendement, tout en limitant les produits toxiques indésirables que l'on pourrait former au cours de la réaction.

\subsection{Conditions expérimentales de la réaction}
Équation de la réaction de synthèse de l'aspirine. Quelques mots à propos de la réaction d'estérification. Choix du solvant : toxicité, propriétés etc... Solutions diverse pour le contrôle de la température qui doit être optimale comme le bain marie par exemple, ou l'utilisation d'un bain réfrigérant pour éviter les vapeurs toxiques (donner les types de réfrigérants usuels : à air, à eau).\\
Il faut trouver un bon compromis entre rendement et vitesse, le temps de réaction représentant un coût évident.\\

La synthèse de l'aspirine est terminée, mais il doit rester de l'anhydride, et de l'acide acétique s'est formé : il faut donc l'éliminer. Cela va être l'occasion de voir les différentes méthodes d'extraction.

\subsection{Choix de la méthode d'extraction}
Notion de solvant extracteur : doit être non miscible avec le solvant réactionnel, et il faut que l'espèce que l'on veut récupérer soit plus soluble dans ce solvant que dans le solvant réactionnel. Une fois le produit extrait dans ce solvant on le récupère en lavant ou en faisant sécher avec un évaporateur rotatif.\\

Ici on va plutôt diminuer la solubilité du produit en baissant la température, le faisant ainsi précipiter, on pourra alors filtrer et récupérer l'acide acétylsalicylique sous la forme  de cristaux. \\
Pour ce faire on met le produit de la synthèse dans un bain de glace tout en le mélangeant à de l'au glacée (plus rapide en direct).\\

Après précipitation on filtre sous Buchner pour récupérer le produit brut. Ce produit comporte cependant des impuretés que l'on va devoir éliminer.

\subsection{Purification}
\begin{itemize}
	\item Première solution : la distillation, schéma et explication.
	\item Pour un solide on va jouer sur le solvant : on prend ici un mélange eau éthanol dans lequel à haute température tout se dissout, tandis que lorsque ce solvant refroidit seule l'aspirine précipite et non l'acide acétique.
\end{itemize}
Il faut ensuite vérifier la pureté du produit final.

\subsection{Analyse}
Il existe diverses méthodes pour analyser le produit final :
\begin{itemize}
	\item On regarde la température de fusion avec un banc Kofler
	\item On fait une CCM
	\item On regarde l'indice optique avec un réfractomètre
\end{itemize}

\section{Sélectivité}
\subsection{Réactifs chimio-sélectifs}
La chimiosélectivité c'est lorsqu'un réactif ne réagit qu'avec un seul groupe caractéristique, dans le cas où on s'attendrait à ce qu'il réagisse avec un autre. Exemple du paracétamol (Term S Nathan ed. 2012 p 502 ; Term S
Hachette ed. 2012 p 492).\\
Lorsque ce n'est pas le cas on peut être amené à protéger une fonction.

\subsection{Protection de fonction}
Exemple de la protection de la fonction cétone. Cette protection doit être active pour un seul groupement en particulier, et être facile à enlever après l'étape concernée.

\subsection{Application à la synthèse peptidique}
Définition de peptide. Quelques exemples. On peut mettre des acides aminés bout à bout grâce à des liaisons peptidiques. Exemple de la Dipeptide Leu-Gly.\\

Importance de la protection de fonction dans le cas des synthèses peptidiques.

\subsection*{Conclusion}
Synthèse de l'exposé. Ouverture à la chimie verte.

\section*{Questions}
Y a t'il toujours un réactif limitant ? Comment le choisit on ?\\
Il est choisi comme tel car c'est le plus cher. Il est limitant car on joue sur l'avancement en mettant les autres réactifs en excès. 

Peut on déplacer l'équilibre autrement ?\\
Oui en éliminant un produit au fur et à mesure.\\

Pouvez vous nous décrire le mécanisme d'estérification (sans utiliser vos notes) ?\\

Pourquoi avoir choisi une anhydride ? Y a t-il d'autres réactifs possibles comme exemples ?\\
Cette réaction est totale. On aurait pu choisir la réaction d'un alcool avec un acide qui n'est  quand à elle pas totale.\\

Pourquoi n'utilise t-on pas le chlorure d'acyle ?\\
Car il n'est pas stable : il est trop réactif. Il vaut donc mieux utiliser l'anhydride.\\

Que se passe t'il si on l'hydrolyse ?\\
Il forme un gaz lacrymogène.\\

Pourquoi ne pas avoir utilisé un chauffe ballon ?\\
Plus pratique pour maintenir la température à 70°C ici.\\

Quel est l'inconvénient du chauffe ballon ?\\
On ne peut alors pas agiter (avec un agitateur magnétique). On doit mettre de la pierre ponce pour réguler l'ébullition et éviter que seul le liquide au fond du ballon ne chauffe en homogénéisant.\\

Le chauffage est il un paramètre seulement cinétique ?\\
Non cela peut être un facteur thermodynamique dans le cas d'une réaction endothermique.\\

Vous dites que le chauffage peut dégrader le produit ... comment ?\\
Ici si on chauffe trop l'acide sulfurique va déshydrater la solution et faire apparaître du carbone.\\

Pouvez vous expliquer le principe de recristallisation ?\\
Il faut faire une une solution saturée à chaud du produit que l'on doit recristalliser puis refroidir.\\

Pour les acides $\alpha-$aminés, ces molécules existent elles sous la forme présentée ?\\
Non elles sont sous la forme $COO^-$.

\section*{Remarques}
Lorsqu'on fait réagir un solide avec liquide il faut commencer par solubiliser un maximum le solide et continuer à agiter (en chauffant) jusqu'à ce que le solide soit complètement dissout.\\

Possibilité d'utiliser de l'acide para toluène sulfonique à la place du sulfurique pour ne pas avoir de problème si on chauffe trop.\\

Il faut mettre un élévateur et préciser son utilité.\\

La température est elle le seul facteur d'impact pour la précipitation dans la synthèse de l'aspirine ?\\
Non on change aussi le solvant, car dans l'eau l'aspirine est peu soluble.\\

Attention il faut faire apparaître tous les produits dans les équations de réaction, et les appeler par leurs noms.\\

\subsubsection*{Plan conseillé :}
Avant l'expérience : je veux faire ça je réfléchis à ce que je peux faire. Quel solvant utiliser, proportion des réactifs .. etc Définir le rendement. Explorer les choix de montages en appuyant surtout sur celui utilisé.\\
J'applique les choix de la partie précédente. Discuter les extractions pendant la phase de précipitation.\\
On fait la purification et l'identification tout en discutant les alternatives.\\
On termine en postulant le choix possible d'un autre réactif, dans le cas de la synthèse du paracétamol (Term S Nathan ed. 2012 p 502 ; Term S
Hachette ed. 2012 p 492), auquel cas ce n'est pas le même groupe qui réagit : ce n'est plus $OH$ mais $NH_3$. Permet de faire la transition vers la sélectivité;




\end{document}