%\documentclass[12pt,prb,aps,epsf]{article}
%\usepackage[utf8]{inputenc}
%\usepackage{amsmath}
%\usepackage{amsfonts}
%\usepackage{amssymb}
%\usepackage{graphicx} 
%\usepackage{latexsym} 
%\usepackage[toc,page]{appendix}
%\usepackage{listings}
%\usepackage{xcolor}
%\usepackage{soul}
%\usepackage[T1]{fontenc}
%\usepackage{amsthm}
%\usepackage{mathtools}
%\usepackage{setspace}
%\usepackage{array,multirow,makecell}
%\usepackage{geometry}
%\usepackage{textcomp}
%\usepackage{float}
%%\usepackage{siunitx}
%\usepackage{cancel}
%%\usepackage{tikz}
%%\usetikzlibrary{calc, shapes, backgrounds, arrows, decorations.pathmorphing, positioning, fit, petri, tikzmark}
%\usepackage{here}
%\usepackage{titlesec}
%%\usepackage{bm}
%\usepackage{bbold}
%
%\geometry{hmargin=2cm,vmargin=2cm}
%
%\begin{document}
%	
%	\title{LC 27 Conversion réciproque d'énergie électrique en énergie chimique}
%	\author{Emmy}
%	
%	\maketitle
%	
%	\tableofcontents
%	
%	\pagebreak
	
	
\subsection{Pré-requis}
Oxydo-réduction, Nernst, thermochimie.

\section{Thermodynamique d'oxydo-réduction}

 \subsection{Électrode : rappel}
 Regarder le \textit{Fosset} \textbf{MPSI} et le \textit{MC Quarrie} de \textbf{Chmie générale}.
 
 \subsection{Point de vue thermodynamique}
 \begin{eqnarray}
 \Delta E = \frac{-\Delta_rG}{n\mathcal{F}}
 \end{eqnarray}
\section{Conversion énergie chimique en énergie électrique}

\subsection{La pile Daniell}
Regarder le chapitre 14 p190 du \textit{Le maréchal} \textbf{Tome 1 Chimie générale}.\\
Présentation plus réalisation expérimentale.

\subsection{Autres exemples}
Regarder le chapitre 14 p200 du \textit{Le maréchal} \textbf{Tome 1 Chimie générale}.

\section{Conversion réciproque}
\subsection{Électrolyseur}
Regarder le chapitre 13 p168 du \textit{Le maréchal} \textbf{Tome 1 Chimie générale}.\\
On transforme cette fois de l'énergie électrique en énergie chimique.
\subsection{Accumulateur}
\textit{Le maréchal} \textbf{Tome 1 Chimie générale} p201.\\
Il faut justifier le choix du plomb.
	
\section*{Questions}



\section*{Remarques}
	
	Il faut être capable de déterminer la capacité d'une pile : nombre d'électrons mis en jeux (= 2*$C_{Cu^{2+}}V$) x leur charge, le tout donné en ampères heure.\\
	Il faut parler de surtension.
	
	
	
	
\end{document}