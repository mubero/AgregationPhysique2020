\documentclass[12pt,prb,aps,epsf]{article}
\usepackage[utf8]{inputenc}
\usepackage{amsmath}
\usepackage{amsfonts}
\usepackage{amssymb}
\usepackage{graphicx} 
\usepackage{latexsym} 
\usepackage[toc,page]{appendix}
\usepackage{listings}
\usepackage{xcolor}
\usepackage{soul}
\usepackage[T1]{fontenc}
\usepackage{amsthm}
\usepackage{mathtools}
\usepackage{setspace}
\usepackage{array,multirow,makecell}
\usepackage{geometry}
\usepackage{textcomp}
\usepackage{float}
%\usepackage{siunitx}
\usepackage{cancel}
%\usepackage{tikz}
%\usetikzlibrary{calc, shapes, backgrounds, arrows, decorations.pathmorphing, positioning, fit, petri, tikzmark}
\usepackage{here}
\usepackage{titlesec}
%\usepackage{bm}
\usepackage{bbold}

\geometry{hmargin=2cm,vmargin=2cm}

\begin{document}
	
	\title{LC 18 Solides cristallins}
	\author{Hallery}
	
	\maketitle
	
	\tableofcontents
	
	\pagebreak
	
	
\subsection{Pré-requis}

\subsection{Introduction}
Les solides peuvent êtres cristallins, il en existe alors différents types : Métalliques, ioniques, moléculaires et covalents.


\section{Cristal}
On constate tout d'abord une structure bien particulière des cristaux à l'échelle macroscopique. En opposition avec les solides amorphes qui n'en ont pas. Le solide est une forme extrêmale de la structure microscopique $\rightarrow$ notion de maille. (ne pas parler de réseau de Bravais : pas au programme). S'attarder sur la maille CFC qui est la seule mentionnée explicitement dans le programme.

\section{Solide métallique et alliage}

Commencer par le cristal métallique : empilement de sphères identiques. Introduire les alliages avec le fait qu'il demeure des interstices.\\
On peut utiliser les modèles moléculaires.\\
On peut aussi utiliser le logiciel "Chim-géné".
	
Ouvrir avec les solides ioniques.

\subsection{Manipulations}
Détermination d'une masse volumique, permet de remonter au paramètre de maille et ainsi au rayon de l'atome métallique, en connaissant le type de maille obtenu par diffraction des rayons X.\\
On commence par mesurer la masse volumique de l'eau : on prend une fiole jaugée que l'on pèse pour obtenir $\rho_{eau}(T_{piece})$ (on peut la comparer avec la valeur du HandBook).\\
On prend ensuite la même fiole, on met du fer (sous forme de poudre) au fond avant de la peser pour déterminer $m_{fer}$. On complète avec de l'eau jusqu'au trait de jauge, et on re-pèse pour obtenir $m_{fer} + m_{eau}$. On en déduit le volume d'eau grâce à la masse volumique et on calcule donc 
\begin{equation}
V_{fer} = V_{fiole} - \frac{m_{eau}}{\rho_{eau}} \Longrightarrow \rho_{fer} = \frac{m_{fer}}{V_{fer}}
\end{equation} 
Sources d'erreurs possibles : bulles d'air dans la poudre, il faut bien sécher la fiole, la poudre peut s'oxyder.
Cristallisation du souffre.
	
\section{Plan possible}	
\subsection{Différents états solides (5 min)}
\subsubsection{Cristal / amorphe}
Qu'est ce qu'un solide cristallins. Différents types de solides cristallins. Opposition avec les solides amorphes.\\

Différentes liaisons et donc différentes propriétés.
\subsubsection{Variétés allotropiques}
\subsection{Modèle du cristal parfait (10 min)}
\subsubsection{Description}
Notion de mailles.\\
Utiliser ChimGene
\subsubsection{Caractéristique}
Compacité, nombre d'atomes par maille, critère de tangence etc...

\subsection{Cristaux métalliques (15 min)}
\subsubsection{Sphères dures}
Empilements AB ou ABC

\subsubsection{CFC du Fer}
\textit{Manip de la densité volumique}

\subsubsection{Alliages}
Il existe des sites dans lesquels on peut insérer d'autres atomes : permet de faire des alliages.
	
\end{document}