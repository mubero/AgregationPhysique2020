\documentclass[12pt,prb,aps,epsf]{report}
\usepackage[utf8]{inputenc}
\usepackage{amsmath}
\usepackage{amsfonts}
\usepackage{amssymb}
\usepackage{graphicx} 
\usepackage{latexsym} 
\usepackage[toc,page]{appendix}
\usepackage{listings}
\usepackage{xcolor}
\usepackage{soul}
\usepackage[T1]{fontenc}
\usepackage{amsthm}
\usepackage{mathtools}
\usepackage{setspace}
\usepackage{array,multirow,makecell}
\usepackage{geometry}
\usepackage{textcomp}
\usepackage{float}
%\usepackage{siunitx}
\usepackage{cancel}
%\usepackage{tikz}
%\usetikzlibrary{calc, shapes, backgrounds, arrows, decorations.pathmorphing, positioning, fit, petri, tikzmark}
\usepackage{here}
\usepackage{titlesec}
%\usepackage{bm}
\usepackage{bbold}

\geometry{hmargin=2cm,vmargin=2cm}

\begin{document}
	
	\title{MP 18 Matériaux demi-conducteurs}
	\author{Etienne}
	
	\maketitle
	
	\tableofcontents
	
	\pagebreak
	
	
\section{Caractéristique d'une thermistance}	
	On va ici tracer la valeur de la résistance $R_{th}$ (mesurée à l'Ohmètre) en fonction de la température, mesurée à l'aide d'une résistance de platine.\\
	On trace ensuite $ln(R_{th})$ en fonction de T afin de valider la pertinence de notre modèle qui prédit 
	\begin{eqnarray}
	\frac{R_{th}(T)}{R_0} = e^{B(\frac{1}{T}-\frac{1}{T_0})}
	\end{eqnarray}
	on peut ensuite faire une modélisation exponentielle de $R_{th}(T)$ afin d'avoir plus de précision sur la détermination des paramètres.\\
	On en déduit finalement les valeurs de $B$ et $R_0$ : 
	\begin{eqnarray}
	R_0 = 10700 \pm 400\, \Omega\\
	B = 4360 \pm 130\, K
	\end{eqnarray}
	On peut en déduire $E_g = h_BB$
	
\section{Sonde à effet Hall}
On trace la tension de Hall (transverse) $U_H$ en fonction de la valeur du champ magnétique au niveau de la plaquette de germanium (on fait varier le champ en variant l'intensité parcourant les deux bobines), en alimentant cette plaquette à courant constant (I=28.646 mA, attention I ne doit pas dépasser 30 mA, dans la pratique on le fixe à l'aide d'une résistance en série du générateur de tension (puisqu'on doit alimenter en 12V)) avec un générateur courant tension.\\
On peut ensuite modéliser $U_H(B)$ avec une droite
\begin{eqnarray}
	U_H =A_H \frac{I}{h} B 
\end{eqnarray}
avec h l'épaisseur de la plaquette (valant 1mm). On en déduit $A_H$ grâce au coefficient directeur obtenu après modélisation. On en déduit ensuite le nombre de porteurs $n_n$ selon 
\begin{eqnarray}
A_H = \frac{1}{n_n\,q_k}
\end{eqnarray}
On peut ainsi en déduire si le semi-conducteur est dopé ou non.

\section{Jonction P-N : électroluminescence}
On a un banc de diodes, de longueurs d'onde différentes, on choisit une intensité (ici 3 mA) et on regarde quelle est la tension seuil $V_S$ associée à cette intensité que l'on conserve constante d'une diode à l'autre. On regarde dans un même temps quel est le $\lambda_{max}$ du spectre de la diode. On trace finalement $\nu_{min} = \frac{1}{\lambda_{max}}$ en fonction de $V_S$ dont on peut déduire $E_g$ pour chaque semi-conducteur de chaque diode selon
\begin{eqnarray}
E_g = h\nu_{min} = eV_S
\end{eqnarray}

\section*{Questions}
Pour la thermistance, pouvez vous représenter le schéma électrique associé à l'un des ohmètres ?\\

Quel est l'ordre de grandeur du courant que l'ohmètre fait passer dans la résistance ?\\ 
Quelque centaines de micro ampères.\\

Pourquoi ne peut on utiliser un courant plus grand ?\\

Car cela provoquerait un auto-échauffement de la résistance par effet Joule.\\

Pensez vous que la résistance des fils parasite de manière conséquente la mesure de la résistance ?\\

Connaissez vous l'ordre de grandeur de la résistance d'un fil de cuivre de 1m ?\\
Quelques Ohm.\\

Y aurait il un moyen de réduire l'incertitude liée à la mesure de température via la résistance de platine ?\\
Il faudrait étalonner la résistance de platine, afin de s'affranchir de la nécessité de lire le tableau de valeurs R(T) donné par le constructeur.\\

Serait il possible, cet étalonnage étant compliqué, d'utiliser l'expression théorique de $R_{pt}(T)$ ? Cela reviendrait à considérer au finale quelle incertitude ?\\
Oui, celle due au ohmètre.\\

Pourquoi la mobilité augmente t-elle avec la température ?\\

Quelle est la différence de comportement entre un semi-conducteur et un conducteur ?\\

Pourquoi, dans la pratique, utilise t-on une thermistance ? Une sonde de platine ?\\

Pour la sonde à effet Hall, quelles sont les précautions à prendre concernant l'établissement du champ magnétique ?\\

Quelle forme de champ cherche t-on à obtenir ?\\

Quel est l'intérêt d'utiliser 2 bobines ?\\
On double le nombre de spires : plus grand B.\\

Pourquoi existe t-il la possibilité d'ajouter une tension de correction sur la plaquette ?\\
Car si la prise de tension se fait sur deux points légèrement décalés par rapport à l'axe on aura un offset.\\

Pourquoi choisir un entrefer parallélépipèdique pour conduire le champ ?\\
Car on veut obtenir un champ homogène autour de notre plaquette.\\

Que se passe t'il si on fait tourner un des entrefers d'un quart de tour autour de son axe ? L'entrefer est il massif ou feuilleté ?\\

Pourquoi est il feuilleté ? Et comment doit on donc le placer (faut il mettre le feuilletage à la perpendiculaire de celui de l'entrefer de dessous ou non)?\\
pour éviter l'échauffement dû aux courants de Foucault.\\

Comment fonctionne le Teslamètre ?\\
Grâce à l'effet Hall.\\

Est ce un avantage pour un capteur d'avoir une caractéristique linéaire ? \\

Manip surprise : pouvez vous mesurer la résistance de la plaquette ?\\
il faut la retirer du circuit puis mesurer la résistance avec un ohmètre équipé de fil que l'on va aller coller de part et d'autre de la plaquette. On obtient 30 $\Omega$.\\ Peut on en déduire la mobilité  $\mu$ ?\\
Oui selon
\begin{eqnarray}
R = \frac{\rho\, L}{l\,h} = \frac{L}{l\,h\,e\,n_n\,\mu} \Rightarrow \mu = \frac{L}{l\,h\,e\,n_n\,R} = 0.27\,m^2\,V^{-1}\,s^{-1}
\end{eqnarray}
\\
Pour mesurer les $\lambda_{max}$ quel matériel avez vous utilisé ?\\
Un spectromètre USB.\\

Comment fonctionne cet appareil ?\\
On disperse la lumière, et on met ensuite une barette de CCD pour chaque angle.\\

Le photorécepteur a t'il sa propre sensibilité ?\\

Comment avez vous donc fait pour l'étalonner ?\\

Quel critère avez vous utilisé pour identifier le $\lambda_{max}$ pour un spectre donné ?\\

Pouvez vous nous donner le schéma électrique associé au montage avec les DELL ?\\

Pourquoi conserve t-on un le même courant pour toutes les DELL ?\\

Comment pourrait on améliorer le protocole ?\\
On pourrait caractériser chacune des DELL et ainsi identifier précisément la tension seuil associée à chacune des diodes.


\section*{Remarques}
Mettre plutôt les schémas électriques des appareils que leurs noms sur les schémas de manip.\\
Il faut bien expliciter les différentes utilisation des semi conducteurs en introduction.\\
On aurait pu mesurer le temps de réaction d'une photodiode, celles ci étant notamment utilisées dans les fibres optiques.
	
\end{document}