\documentclass[12pt,prb,aps,epsf]{article}
\usepackage[utf8]{inputenc}
\usepackage{amsmath}
\usepackage{amsfonts}
\usepackage{amssymb}
\usepackage{graphicx} 
\usepackage{latexsym} 
\usepackage[toc,page]{appendix}
\usepackage{listings}
\usepackage{xcolor}
\usepackage{soul}
\usepackage[T1]{fontenc}
\usepackage{amsthm}
\usepackage{mathtools}
\usepackage{setspace}
\usepackage{array,multirow,makecell}
\usepackage{geometry}
\usepackage{textcomp}
\usepackage{float}
%\usepackage{siunitx}
\usepackage{cancel}
%\usepackage{tikz}
%\usetikzlibrary{calc, shapes, backgrounds, arrows, decorations.pathmorphing, positioning, fit, petri, tikzmark}
\usepackage{here}
\usepackage{titlesec}
%\usepackage{bm}
\usepackage{bbold}

\geometry{hmargin=2cm,vmargin=2cm}

\begin{document}
	
	\title{LC 28 Solubilité}
	\author{Naïmo Davier}
	\date{Agrégation 2019}
	\maketitle
	
	\tableofcontents
	
	\pagebreak
	
	
\section{Introduction}
Niveau CPGE première année.\\

Pré-requis : Notion d'équilibre chimique, ions.\\

On peut constater que lorsqu'on met trop de sucre dans l'eau on finit par ne plus pouvoir le dissoudre : on voit alors apparaître un précipité.
Définir la notion de précipité, illustrer avec AgCl.

\section{Précipitation et solubilité}
\subsection{Produit de solubilité}
\textit{Chmie tout en un PCSI} de \textbf{Fosset} p871\\

Donner la définition du produit de solubilité (solide ionique ou non), illustrer avec la réaction 
\begin{eqnarray}
AgCl_{(s)} = Ag^+_{(aq)} + Cl^-_{(aq)}
\end{eqnarray}
Préciser que le solide n'existe pas nécessairement (le préciser en introduisant le quotient réactionnel).\\

\textbf{Manip} : Détermination d'un $K_s$ par conductimétrie \textit{Le maréchal tome 1 : chimie générale p160}.

\subsection{Domaine d'existence}

Introduire alors la notion de domaine d'existence pour un précipité, faire le diagramme $AgCl$.

\subsection{Solubilité et effet d'ion commun}
\textit{Chmie tout en un PCSI} de \textbf{Fosset} p876-878\\

Définir la notion de solubilité. Expliciter l'effet d'ion commun.\\

\textbf{Application} : on peut faire précipiter les ions sulfate et les ions fluorure, nocifs pour la santé (le sulfate en trop grande concentration provoque de forces diarrhées, et le fluor en trop grande quantité peut provoquer ed l'ostéoporose et abimer les muscles et les reins), en mettant des ions $Ca^{2+}$ en excès dans les égouts 
\begin{eqnarray}
SO_4^{2-} + Ca^{2+} + 2 H_2O_{(l)}= (CaSO_{4}, 2H_O )_{(s)}\\
2F^-+Ca^{2+} = CaF_{2(s)}
\end{eqnarray}

\subsection{Facteurs influençant la solubilité}
En plus de l'effet d'ion commun que l'on vient de commenter, on peut citer l'influence de deux autres facteurs sur la solubilité :\\

Influence de la température : manip de la pluie d'or dans \textbf{Le maréchal} \textit{tome 1 : Chimie générale}\\

Influence du PH : compétition avec la complexation :\\
faire le cas de l'argent dans \textit{Des expériences de la famille acide-base} de \textbf{Cachau-Herreillat} p96

\section{Utilisations}
On peut utiliser les notions introduites à divers effets :

\subsection{Dosage par précipitation}
Dosage par la méthode de Mohr \textbf{Le maréchal} \textit{tome 1 : Chimie générale}

\subsection{Caractérisation des ions}
On peut caractériser les ions en fonction de la couleur et des propriétés des précipités qu'ils forment.\\

\textit{Des expériences de la famille acide-base} de \textbf{Cachau-Herreillat} p96 :\\ comparer les précipités de l'argent, du zinc, du cuivre et du fer formés avec des ions hydroxyde.

\subsection{Précipitation des métaux : procédé Bayer}
\textit{Term S spécialité hachette 2012 p135\\ ou Term S spécialié Belin ed 2002 p197\\ ou 100 manipulations de chimie générale et analytique} de \textbf{J.Mesplède}

\section{Conclusion}
On a vu un nouveau type d'équilibre et qu'on pouvait se servir du phénomène dans différents cas, et notamment dans l'industrie pour traiter les ions qui polluent les eaux en sortie d'usine, et pour séparer les ions afin d'isoler ceux qui nous intéressent des autres.

\section*{Questions}
Qu'est ce que la recristallisation ?\\

De quoi dépend Ks ?\\

C'est quoi $Q_r$ ?\\

dissolutions exothermique ou endothermique? comment savoir? \\
une fois ions sépares, il se passe quoi? ils se solvatent.\\

pk on fait deux mesures de potentiel dans l'expérience pour calculer le Ks?\\

intérêt de prendre une solution de concentration inconnue? \\

dosage méthode de Mohr, principe?\\

\section*{Remarques}
Attention pendant l'intro à ne pas confondre solubilisation et précipitation.\\
Il faut suivre à la lettre le programme de prépa ici. Les manipulations servent ici à illustrer la leçon.\\


\end{document}