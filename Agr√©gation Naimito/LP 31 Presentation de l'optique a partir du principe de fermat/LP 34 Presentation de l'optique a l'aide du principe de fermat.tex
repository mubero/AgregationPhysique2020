\documentclass[12pt,prb,aps,epsf]{report}
\usepackage[utf8]{inputenc}
\usepackage{amsmath}
\usepackage{amsfonts}
\usepackage{amssymb}
\usepackage{graphicx} 
\usepackage{latexsym} 
\usepackage[toc,page]{appendix}
\usepackage{listings}
\usepackage{xcolor}
\usepackage{soul}
\usepackage[T1]{fontenc}
\usepackage{amsthm}
\usepackage{mathtools}
\usepackage{setspace}
\usepackage{array,multirow,makecell}
\usepackage{geometry}
\usepackage{textcomp}
\usepackage{float}
\usepackage{cancel}
\usepackage{here}
\usepackage{titlesec}
\usepackage{bbold}

\geometry{hmargin=2cm,vmargin=2cm}

\begin{document}
	
	\title{LP 31 Présentation de l'optique géométrique à l'aide du principe de Fermat}
	\author{Emy}
	
	\maketitle
	
	\tableofcontents
	
	\pagebreak
	
\section{introduction}
Annonce du plan et du résultat final: expliquer comment nos perceptions peuvent êtres faussées du à la trajectoire non nécessairement rectiligne des rayons lumineux.

\section{Principe de Fermat}
\subsection{Enoncé}
Enoncé historique du principe de Fermat, puis formulation moderne à l'aide du schéma optique. Cet énoncé moderne est moins restrictif et prend donc en compte certains cas particuliers.
\paragraph{Chemin optique}
schéma correspondant : 2 points A et B reliés par une trajectoire $\mathcal{C}$\\
Le chemin optique est def comme 
\begin{eqnarray}
L_{AB} = \int_A^BndS
\end{eqnarray}
où n est l'indice de réfraction du milieu. Le chemin optique est, selon le principe de Fermat, stationnaire.\\
Schéma correspondant: 2 courbes relient deux points A et B, $\mathcal{C}$ et $\mathcal{C}'$, alors $\delta L = L-L' = 0$ au premier ordre en $||\vec{\delta M}||_{max}$ si on note $\vec{\delta M}$ le vecteur différence entre les deux trajectoires.
\subsection{Premières conséquences}
\paragraph{Cas d'un milieu homogène}
\begin{eqnarray}
L_{AB} = \int_A^BndS = n\int_A^BdS = nAB
\end{eqnarray}
minimum si la trajectoire du rayon est rectiligne.
\paragraph{Retour inverse}
\begin{eqnarray}
L_{AB} =\int_A^BndS = \int_B^An(-dS) = n\int_B^AdS'\; FAUX\;!!!
\end{eqnarray}
car dans ce cas on a un truc positif égal à un truc négatif.
\begin{eqnarray}
\Rightarrow L_{AB} =\int_A^BndS = n\int_B^AdS'
\end{eqnarray}
Un rayon emprunte donc la même trajectoire dans un sens que dans l'autre.

\section{Lois de Snell-Descartes}
schéma rayon passant par un dioptre séparant deux milieux d'indices $n_1$ et $n_2$, le rayon allant de A vers B et passant à travers le dioptre au point I, avec un angle incident $i$.\\
\begin{eqnarray}
L_{AB} &=& L_{AI}+L_{IB}\\
\delta L_{AB} &=& n_1\vec{u}_1.\vec{\delta I} -\vec{u}_2.\vec{\delta I} = (n_1\vec{u}_1-\vec{u}_2).\vec{\delta I} = 0
\end{eqnarray}
On a donc $n_1\vec{u}_1-\vec{u}_2\perp \vec{\delta I}=\vec{II}'$. et donc $n_1\vec{u}_1-n_2\vec{u}_2= \alpha  \vec{N}$.

\subsection{Lois de réfraction}
\begin{eqnarray}
\vec{u}_2 = \frac{n_2\vec{u}_1+\alpha\vec{N}}{n_2} \Rightarrow \vec{u}_2 \in (\vec{u}_1,\vec{N})
\end{eqnarray}
et 
\begin{eqnarray}
n_1\vec{u}_1-n_2\vec{u}_2= \alpha  \vec{N}\\
n_1\vec{u}_1\times\vec{N}-n_2\vec{u}_2\times\vec{N}= 0\\
n_1 \mathrm{sin}(i) = n_2 \mathrm{sin}(r)
\end{eqnarray}

\subsection{Lois de réflection}
Schéma\\
\begin{eqnarray}
\vec{u}_2 = \frac{n_2\vec{u}_1+\alpha\vec{N}}{n_1} = \vec{u}_1 + \frac{\alpha \vec{N}}{n_1} \Rightarrow \vec{u}_2 \in (\vec{u}_1,\vec{N})
\end{eqnarray}
et 
\begin{eqnarray}
n_1\vec{u}_1-n_1\vec{u}_2= \alpha  \vec{N}\\
n_1\vec{u}_1\times\vec{N}-n_1\vec{u}_2\times\vec{N}= 0\\
 \mathrm{sin}(i) = \mathrm{sin}(r)\\
 i=r
\end{eqnarray}
expérience avec demi cylindre de plexiglas sur un disque gradué (ici photos au vidéoprojecteur pour que l'expérience soit bien claire pour le public) pour mesurer 
\begin{eqnarray}
r(i) =\arcsin\left(\frac{n_1\mathrm{sin}(i)}{n_2}\right)\simeq\arcsin\left(\frac{\mathrm{sin}(i)}{n_2}\right) 
\end{eqnarray}  si on travaille dans l'air. Avec les deux cas $n_1>n_2$ et $n_1<n_2$, et mise en évidence de l'angle limite dans le cas $n_1>n_2$, en effet on a 
\begin{eqnarray}
n_1\sin(i) = n_2\sin(r) \Rightarrow -1 < \sin(r) < 1 \Rightarrow -\frac{n_1}{n_2} < \sin(i) < \frac{n_1}{n_2}
\end{eqnarray}
Il existe donc un angle d'incidence limite au delà duquel il n'existe plus de rayon réfracté : il ya alors réflexion totale.

\section{Equations du rayon dans un milieu non homogène}
Schéma : modélisation d'un milieu non homogène via une stratification de couche avec une différence d'indice $\delta n$ d'une couche à l'autre. Par conséquent on a $\vec{grad}\,n$ qui est normal aux surfaces équi-indice. On a donc, en partant de 
\begin{eqnarray}
n_1\vec{u}_1-n_1\vec{u}_2= \alpha  \vec{N},\\
n\vec{u}-(n+dn)(\vec{u}+\vec{du})= \alpha  \vec{N}\; \mathrm{ici}\\
\mathrm{c.a.d}\;\;d(n\vec{u}) = \alpha\vec{N} \\\
\Rightarrow d(n\vec{u}) = \beta \vec{grad}\,n\\
\Rightarrow (n\vec{du} + dn\vec{u}).\vec{u} = \beta \vec{grad}\,n.\vec{u}\\
\Rightarrow dn = \beta \vec{grad}\,n.\vec{u} =  \beta \frac{dn}{ds}
\end{eqnarray}
Et finalement on obtient 
\begin{eqnarray}
\frac{dn\vec{u}}{ds} = \vec{grad}\,n,
\end{eqnarray}
c'est le théorème fondamental de l'optique.
\subsection{Application au phénomène su mirage}
Schéma (plus calculs).\\
Illustration avec laser traversant cuve avec gradient d'indice dû à un gradient de salinité de l'eau dans la cuve.

\section{Conclusion}
Récapitulatif des principaux résultats, + application au calcul du trajet de la lumière dans tout milieu donc on connait l'indice optique. Analogie avec la lentille gravitationnelle.

\section{Questions}
Concernant les schémas des mirages : expliquer mieux le schéma.\\

Pourquoi l'indice de réfraction dépend t'il de la densité du milieu ?\\

Analogie avec les lentilles gravitationnelles :  l'indice optique du vide intersidéral varie t'il ? \\

Cuve avec sel : l'eau de la cuve est elle saturée en sel ? le choix du sel pour créer une variation d'indice optique est il pertinent ?\\

Pourquoi, pour cette expérience, ajoute t-on de la fluorescéine ?\\

Et pourquoi, dans ce cas, voit on le trajet des rayons émis par le laser ?


\section{Commentaires et réponses aux questions}
Regarder formule de Clausius Mosoti : lien entre l'indice optique et la densité du milieu. Regarder aussi le modèle de Drude-Lorrentz qui exprime comment le champ électrique porté par une onde lumineuse polarise le milieu de propagation. \\
La fluorescéine ne peut s'utiliser qu'avec un laser vert (et non rouge), car elle réémet dans le jaune (phénomène d'absorption-émission : fluorescence) : les photons rouges ne sont pas assez énergitiques pour l'exciter.\\
Message majeur de cette leçon : les résultats ici donnés ne relèvent pas d'une analyse différentielle mais d'un principe variationnel.\\
Il faut aller plus vite : il faut faire + court et dire plus de choses.\\
Il faut définir la notion de rayon lumineux, et l'optique géométrique !!!!\\
Expliciter pourquoi utiliser le principe de Fermat et pas les équations de Maxwell : on veut montrer l'optique sous un angle variationnel. Peut être faire l'équation de Eikonal en intro pour montrer que l'on peut retrouver ça à partir de Maxwell.\\
Ponctuer la leçon, tenter de ne pas être monotone, il faut presser les calculs : les expliquer, faire des analogies (ici avec les potentiels en mécanique pour l'indice), des illustrations..etc Il faut mener les calculs et pas juste les présenter, dos au jury, sans parler.\\
Conseil écrire EN GROS les calculs sur des feuilles de note à part pour pouvoir n'y jeter qu'un oeil et ne pas les lire en les gardant en mains.\\
Utiliser des termes qui souligne l'intérêt ce que l'on est en train de faire.\\
Attention aussi à la rigueur mathématique : il faut être irréprochable sur les notations, et sur l'énonciation des grandeurs en jeu.\\
L'introduction doit partir d'un physique très générale, tout comme la conclusion doit ouvrir à d'autres domaines.\\
Conclure sur les manip, surtout après le calcul d'incertitude dont on doit commenter le résultat.\\
Bonne application de la réflexion totale : fibres optiques à saut d'indice. et rq : réflexion totale $\neq$ bonne réflexion : elle est TOTALE !! la réfraction est ici interdite par la physique. c'est grâce à ça que le taux d'atténuation des fibres est très faible.\\


\end{document}