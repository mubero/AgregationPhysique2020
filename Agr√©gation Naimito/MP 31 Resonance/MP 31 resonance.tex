\documentclass[12pt,prb,aps,epsf]{report}
\usepackage[utf8]{inputenc}
\usepackage{amsmath}
\usepackage{amsfonts}
\usepackage{amssymb}
\usepackage{graphicx} 
\usepackage{latexsym} 
\usepackage[toc,page]{appendix}
\usepackage{listings}
\usepackage{xcolor}
\usepackage{soul}
\usepackage[T1]{fontenc}
\usepackage{amsthm}
\usepackage{mathtools}
\usepackage{setspace}
\usepackage{array,multirow,makecell}
\usepackage{geometry}
\usepackage{textcomp}
\usepackage{float}
%\usepackage{siunitx}
\usepackage{cancel}
%\usepackage{tikz}
%\usetikzlibrary{calc, shapes, backgrounds, arrows, decorations.pathmorphing, positioning, fit, petri, tikzmark}
\usepackage{here}
\usepackage{titlesec}
%\usepackage{bm}
\usepackage{bbold}

\geometry{hmargin=2cm,vmargin=2cm}

\begin{document}
	
	\title{MP 31 Résonance}
	\author{Clément}
	
	\maketitle
	
	\tableofcontents
	
	\pagebreak
	
	
\section{Introduction}

\section{Corde de Melde}

Corde excitée par un vibrateur et oscillant donc à une fréquence donnée.\\
Vibrateur : Imax = 1A et Vmax = ?\\
On va appliquer un signal sinusoïdal dd'amplitude 3V, avec des masselottes de 200g et 1g.
Lorsque l'amplitude de la vibration de la corde est max on relève la fréquence su signal d'entrée à l'oscillo, et on note ainsi la fréquence associée à chacun des modes. On peut ensuite modéliser f(n) par une droite (n étant le n° du mode). On en déduit que $f = \alpha n$, ce qui est cohérent puisque l'on peut montrer analytiquement que $\alpha =\frac{c}{2l}$, avec $c$ la célérité de la corde. On a aussi $c=\sqrt{\frac{T}{\mu}}=\sqrt{\frac{m*g}{m_c/l}}$, on peu regarder si les résultats sont cohérents entre eux.\\
Incertitudes :
\begin{eqnarray}
\left(\frac{u_c}{c}\right)^2 &=& \left(\frac{u_{\alpha}}{\alpha}\right)^2+\left(\frac{u_l}{l}\right)^2\\
&=& \frac{1}{2}\left( \left(\frac{u_{m_c}}{m_c}\right)^2+\left(\frac{u_m}{m}\right)^2+\left(\frac{u_l}{l}\right)^2\right)
\end{eqnarray}

\section{Circuit RLC en Série}
schéma\\
\subsubsection{Mesure en intensité}
on regarde $U_R(t)$, et on observe quelle est la réponse en intensité selon la fréquence du signal d'entrée, pour quelques valeurs de R données, on remarque que l'intensité max décroît avec R.\\
\begin{eqnarray}
Q&=&\frac{L\omega_0}{R} \; \Rightarrow \; \left(\frac{u_Q}{Q}\right)^2 = \left(\frac{u_{\omega_0}}{\omega_0}\right)^2+\left(\frac{u_L}{L}\right)^2+\left(\frac{u_R}{R}\right)^2\\
&=& \frac{f_0}{f_{c_1}f_{c_2}}
\end{eqnarray}
où les $f_{c_i}$ sont les deux fréquences de coupure à -3dB autour du pic de résonance.
\subsubsection{Mesure en énergie}
On regarde ici E(f).
\section{Questions}

$i)$ Pourquoi as tu branché le GBF au vibrateur de cette manière ?\\
$\rightarrow$ c'est dû à la résist interne du vibrateur, on veut observer une tension précise et on va donc se heurter à la notion d'adaptation d'impédance.\\

$ii)$ Comment fonctionne le trigger ?\\
$\rightarrow$ balayage selon x, et selon y (en l'occurrence selon t et selon V). Trigger c'est déterminer le temps d'attente entre deux affichages du signal, en fixant la tension à laquelle l'affichage "démarre".\\
Rq: ne pas appuyer sur "autoscale", il faut régler manuellement pour bien comprendre et être capable d'anticiper.\\

$iii)$ Comment as tu choisit ta méthode de mesure de la fréquence pour la corde de Melde ?\\
$\rightarrow$ on mesure plusieurs périodes.\\

$iv)$ As tu pris en compte l'incertitude liée à la mesure de la fréquence dans le calcul de c ?\\
$\rightarrow$ Oui c'est pris en compte dans la modélisation de $f(n)$ et se retrouvera donc sur l'incertitude de $\alpha$.\\

$v)$ Comment régressi calcule lo modèle de la droite ? Quel est le principe de l'algorithme ?\\
$\rightarrow$ Méthode des moindres carrés : on minimise la somme des carrés de l'écart de chacun des points au modèle.\\

$vi)$ Comment as tu choisi les paramètres d'acquisition sous latis pro pour la caractérisation du RLC ?\\
$\rightarrow$ On prend 50 points par période pour avoir une bonne résolution sans problèmes liés au quantum + on règle la plage en fonction de l'amplitude max du signal.\\

\paragraph{Montage surprise :} caractérisation d'un RLC aux bornes de R.\\
On sait, en résolvant les équa diff que le signal $U_R(t)$ va être une sinusoïde de période $\omega_0=\sqrt{\frac{1}{LC}}$ modulée par une exp décroissante de temps caract $\tau = L/R$. On choisit donc la valeur des composants de sorte à observer plusieurs oscillations : R doit être suffisamment petit et C suffisamment grand. On peut ensuite observer à l'oscillo la réponse à une échelon (ou on utilise un interrupteur).\\
Rq: on aura une intensité nulle si $V_e(t)=0$, mais aussi lorsque $V_e(t)\neq0$ dès que le condensateur est chargé.\\

\section{Remarques générales}
Possibilité de parler de résonance paramétrique, demande un peu plus de maitrise.\\
Pour le RLC il faut discuter le lien entre aspect du diagramme de Bode et valeur de la résistance.\\
Possibilité de montage mécanique(analogie avec elec possible) : masse(L) + ressort(k=1/L) + amortisseur(R).\\
On peut tracer la puissance au niveau de la résistance en utilisant un multiplieur faisant $V_e x V_R$ (avec une résistance de $1\Omega$ par exemple). On trace ensuite $P(\omega)$.\\
La notion d'énergie associée à la résonance est pertinente car transposable à pas mal de domaines de la physique : elec, méca, MQ...
\end{document}
