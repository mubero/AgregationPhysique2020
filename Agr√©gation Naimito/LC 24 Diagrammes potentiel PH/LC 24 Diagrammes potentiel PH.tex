\documentclass[12pt,prb,aps,epsf]{article}
\usepackage[utf8]{inputenc}
\usepackage{amsmath}
\usepackage{amsfonts}
\usepackage{amssymb}
\usepackage{graphicx} 
\usepackage{latexsym} 
\usepackage[toc,page]{appendix}
%\usepackage{listings}
\usepackage{xcolor}
\usepackage{soul}
\usepackage[T1]{fontenc}
\usepackage{amsthm}
\usepackage{mathtools}
\usepackage{setspace}
\usepackage{array,multirow,makecell}
\usepackage{geometry}
\usepackage{textcomp}
\usepackage{float}
\usepackage{cancel}
\usepackage{here}
\usepackage{titlesec}
\usepackage{bbold}

\geometry{hmargin=2cm,vmargin=2cm}

\begin{document}
	
	\title{LC 24 Diagrammes potentiel PH}
	\author{Naïmo Davier}
	
	\maketitle
	
	\tableofcontents
	
	\pagebreak
	
\subsection{Introduction et pré-requis}
Niveau 1ère année S2 PCSI.\\
Réactions acido basiques et RedOx, solubilité, relation de Nernst, construction des diagrammes E-PH.\\

Manip d'introduction : montrer que le PH influence sur le pouvoir oxydant du permanganate, \textit{L'oxydoréduction} de \textbf{J. Sarrazin}.\\

Cette leçon va permettre d'expliquer cette observation et de généraliser cette étude avec un outils adapté : le diagramme potentiel PH.

\section{Diagrammes potentiels PH}
\subsection{Rappels}
\textit{Chimie tout en un PCSI} de \textbf{Bruno Fosset} p1019.\\

Reconstruire rapidement le diagramme potentiel PH de l'eau, et s'en servir pour redéfinir la concentration/pression de tracé, énoncer que c'est construit à une température donnée, et enfin rappeler la notion de prédominance et d'existence. 

\subsection{Utilisations}
\subsubsection{Explication de la manip d'intro}
Expliquer l'observation faite en introduction : \textit{Chimie tout en un PCSI} de \textbf{Bruno Fosset} p1023.\\

\subsubsection{Diagramme potentiel PH du Fer}
Différentes manipulations illustrant la lecture des diagrammes.\\
\textbf{Mesplède Randon} \textit{chimie générale} p220

\section{Application : méthode de Winkler}
Théorie et diagramme E-PH :
\textit{Chimie tout en un PCSI} de \textbf{Bruno Fosset} p1042.\\
Manip :
\textbf{Le Maréchal} \textit{Tome 1 Chimie générale} p77.

\section{Application à l'hydrométallurgie}

\subsection{L'hydrométallurgie}
Faire une rapide introduction à l'hydrométallurgie : page wikipedia bien faite.

\subsection{Illustration pratique}
Manip :
\textit{Term S Spécialité Belin ed 2002} 188. (et BUP 770 janvier 1995 p111)

\section{Conclusion}
On a pu aujourd'hui vois comment utiliser cet outil pour prévoir les équations RedOx qui vont se produire du point de vue thermodynamique. Cet outils reste tout de même limité car il ne nous dit rien quand à la cinétique et quand aux réactions que l'on peut forcer telles que les hydrolyses par exemple.

	
\end{document}