\documentclass[12pt,prb,aps,epsf]{article}
\usepackage[utf8]{inputenc}
\usepackage{amsmath}
\usepackage{amsfonts}
\usepackage{amssymb}
\usepackage{graphicx} 
\usepackage{latexsym} 
\usepackage[toc,page]{appendix}
%\usepackage{listings}
\usepackage{xcolor}
\usepackage{soul}
\usepackage[T1]{fontenc}
\usepackage{amsthm}
\usepackage{mathtools}
\usepackage{setspace}
\usepackage{array,multirow,makecell}
\usepackage{geometry}
\usepackage{textcomp}
\usepackage{float}
\usepackage{cancel}
\usepackage{here}
\usepackage{titlesec}
\usepackage{bbold}

\geometry{hmargin=2cm,vmargin=2cm}

\begin{document}
	
	\title{LP 19 Bilans thermiques : flux conductifs, convectifs et radiatifs}
	\author{Clément}
	\date{Agrégation 2019}
	
	\maketitle
	
	\tableofcontents
	
	\pagebreak
	
\subsection{Introduction}
Nécessite de comprendre les transferts thermiques pour optimiser les systèmes thermiques : moteurs, isolation..etc

\section{Rayonnement et convection}
\subsection{Rayonnement}
La matière émet et absorbe de la chaleur, il y a de l'émission due à l'agitation thermique, et de l'absorption et augmente cette agitation thermique. Pour le rayonnement on va s'intéresser principalement au corps noir, définition.\\
Loi de Planck
\begin{eqnarray}
E_{0\lambda} = \frac{2\pi c_1}{\lambda^5(e^{c_2/\lambda T}-1)}
\end{eqnarray}
on dérive par rapport à $\lambda$ pour trouver le maximum d'émission ou d'absorption. Cas du soleil.\\

Loi de Stephan : on a pour la densité de puissance $E_0 = \int_0^{\infty}E_{0\lambda}d\lambda = \sigma T^4$.

\subsection{Conduction thermique}
Hypothèse des milieux continus (on travaille à des échelles grandes devant la taille des molécules et du libre parcours moyen). \\
Loi de Fourrier 
\begin{eqnarray}
\vec{j} = -\lambda \vec{\nabla}T
\end{eqnarray}
On définit alors le flux de chaleur comme
\begin{eqnarray}
\Phi = \iint _S \vec{j}.\vec{n}dS
\end{eqnarray}
on a alors 
\begin{eqnarray}
Q = \int_0^t \Phi dt
\end{eqnarray}

\section{Convection thermique}
\subsection{Première approche}
\subsection{Analyse dimensionnelle}
\begin{itemize}
	\item Transfert de quantité de mouvement
	\item Transfert convectif d'enthalpie
	\item Transfert conductif transverse
\end{itemize}

Nombres de Prandtl, Nusselt etc...

\section{Application : chauffage ...}


\section*{Questions}
D'où vient l'expression de l'expression du nombre de Nusselt en fonction des nombres de Prandtl et de Reynolds ?\\

Des exemples où un bilan radiatif permet de déterminer la température d'un corps ?\\
Les planètes.

\section{Plan conseillé}
\subsubsection{Omniprésence de ces phénomènes}
Présence quotidienne : terre chauffée par le soleil et par son centre via le manteau, chauffage dans une maison etc...
\subsection{Rappels sur les flux conductifs et radiatifs}
\subsection{Convection}
Manip des deux erlen : l'un rempli d'eau tiède colorée, l'autre d'eau froide sans colorant, on pose le premier sur le secont et on observe.

\end{document}