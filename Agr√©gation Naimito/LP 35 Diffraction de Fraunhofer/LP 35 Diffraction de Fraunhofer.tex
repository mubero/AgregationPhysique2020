\documentclass[12pt,prb,aps,epsf]{report}
\usepackage[utf8]{inputenc}
\usepackage{amsmath}
\usepackage{amsfonts}
\usepackage{amssymb}
\usepackage{graphicx} 
\usepackage{latexsym} 
\usepackage[toc,page]{appendix}
\usepackage{listings}
\usepackage{xcolor}
\usepackage{soul}
\usepackage[T1]{fontenc}
\usepackage{amsthm}
\usepackage{mathtools}
\usepackage{setspace}
\usepackage{array,multirow,makecell}
\usepackage{geometry}
\usepackage{textcomp}
\usepackage{float}
%\usepackage{siunitx}
\usepackage{cancel}
%\usepackage{tikz}
%\usetikzlibrary{calc, shapes, backgrounds, arrows, decorations.pathmorphing, positioning, fit, petri, tikzmark}
\usepackage{here}
\usepackage{titlesec}
%\usepackage{bm}
\usepackage{bbold}

\geometry{hmargin=2cm,vmargin=2cm}

\begin{document}
	
	\title{LP 35 Diffraction de Fraunhofer}
	\author{Etienne}
	
	\maketitle
	
	\tableofcontents
	
	\pagebreak
	
	
\subsubsection{prérequis}
Transformée de fourrier\\
Optique géométrique\\
interférences

\section{Introduction}
\paragraph{Définition} Tout ce qui ne relève ni de réfractions ni de réflexions.\\
\paragraph{Expérience} Diffraction par une fente, inexplicable par une théorie corpusculaire de la matière.

\section{Théorie scalaire de la diffraction}
\subsection{Principe d'Huygens-Fresnel}
Introduction historique.\\
Énoncé du principe d'Huygens, apport de Fresnel, formulation :
\begin{eqnarray}
\phi(M) = \alpha \int_{\Sigma} d^2P\;\phi_i(P)\, \frac{e^{ikPM}}{PM}\;\mathrm{avec}\; \alpha = \frac{-iQ}{\lambda}
\end{eqnarray}
\subsection{Diffraction en champ proche}
On va définir le flux incident au point P $\phi(P)$ à partir de la transmittance $\tau$ et du flux incident avant la surface $\phi_0$ : $\phi_i(P)=\tau (P)\phi_0$.\\
On peut ensuite exprimer la distance PM en fonction de coordonnées judicieusement choisies dans le plan de l'objet diffractant, et dans le plan d'observation : $P(X,Y)$ et $M(x,y)$. En définissant D comme étant la distance entre ces deux plans on obtient 
\begin{eqnarray}
PM \simeq D + \frac{(x-X)^2+(y-Y)^2}{2D}
\end{eqnarray}
On en déduit 
\begin{eqnarray}
\phi(M) \simeq \frac{\alpha}{D} \phi_0 \int_{\mathcal{S}} dXdY\;\tau(X,Y)\, e^{ik (D + \frac{(x-X)^2+(y-Y)^2}{2D}) }
\end{eqnarray}
On va chercher à se placer dans des configuration permettant de calculer cette intégrale compliquée, qui nécessite l'utilisation de fonctions spéciales dans le cas général.
\section{Diffraction en champ lointain}
\subsection{Approximation de Fraunhofer}
On se place dans des conditions telles que 
\begin{eqnarray}
D \gg \frac{R^2}{\lambda}
\end{eqnarray}
On peut utiliser une lentille convergente afin de se placer dans le cas $D \rightarrow \infty$, schéma. On passe alors de $kPM$ à $k_0\delta$, et on montre, en utilisant le principe de retour inverse de la lumière et le fait qu'une lentille cvg conjugue une onde plane avec une onde sphérique que 
\begin{eqnarray}
\delta = -\frac{xX+yY}{f_i}
\end{eqnarray}
Ainsi on obtient 
\begin{eqnarray}
\phi(M) \simeq \frac{\alpha}{D} \phi_0 \int_{\mathcal{S}} dXdY\;\tau(X,Y)\, e^{ik_0 (ux + vy) }
\end{eqnarray}
avec $u = \frac{x}{\lambda f_i}$ et $v = \frac{y}{\lambda f_i}$. On remarque donc que $\phi(M)$ est la TF de la transmittance.

\subsection{Calcul de quelques figures de diffraction}
\subsubsection{Fente rectangulaire}
\begin{eqnarray}
\acute{E}(u,v) = \frac{\acute{E}_0}{\lambda^2D^2}(ab)^2\,sinc^2(ua)\,sinc^2(vb)
\end{eqnarray}
On a donc les extinctions successives décrites par $U_n =\frac{n}{a}$ Projection d'une image représentant ce résultat.
\subsubsection{Ouverture circulaire}
Projection du calcul et de la figure de diffraction obtenue.\\

Question : la diffraction modifie donc elle tout ce qui a été fait en optique géométrique ou ajoute elle seulement quelque chose qui avait été négligé jusque alors ?

\section{Lien entre diffraction et formation des images}
\subsection{Influence de la rotation de l'éclairage}

Schéma. 
\begin{eqnarray}
\delta_{sup} = X\sin(\theta_0)\\
\delta_{tot} = -X\frac{x-x_0}{f_i}
\end{eqnarray}
La figure se forme donc autour de l'image de la source par le système optique objet diffractant + lentille. Cette configuration étant toujours "vérifiée" lorsque l'on conjugue un objet par une lentille, on conjugue donc une tache de diffraction à un point objet.

\section*{Conclusion}
Limites de résolution de tout instrument d'optique due à la diffraction qui "habille" toute image que l'on fait d'un objet en utilisant un système optique.

\section*{Questions}
Dans le montage réalisé pour illustrer les phénomènes modélisés, pourquoi utiliser un objectif de microscope ?\\
Permet de transformer le faisceau laser en un point source. On utilise ensuite une lentille cvg pour éclairer l'objet avec une onde plane, en la plaçant à une distance $f_{lentille}+f_{microscope}$ du microscope.\\

Concernant la diffraction par une fente et un disque : pourquoi parler de fréquence spatiale pour la fente et de rayon (longueur) pour la tache d'airy produite par le trou circulaire ?\\

Pouvez vous justifier la différence entre les figures de diffraction d'un disque ou d'un carré ? Pourquoi celle du cercle est plus étalée ?\\

\section*{Commentaires}
Attention, dans le cas où l'objet diffractant a une taille caractéristique $a$, la taille de la figure est proportionnelle à $\frac{\lambda D}{a}$ OU $\frac{\lambda f_i}{a}$ selon le montage adopté. Si l'objet n'a pas de taille caractéristique, il n'existe pas de diffraction de Fraunhofer, mais l'objet diffracte bel et bien.\\
Mieux d'écrire l'intégrale tout en énonçant le principe d'Huygens-Fresnel.\\
Bien définir la notion de surface d'onde, et ainsi bien définir le domaine d'intégration pour $\phi$.\\
Facteur d'obliquité $Q$?? permet d'exprimer qu'il n'y a pas de rétro-diffusion, et valable pour de petits angles... regarder dans le Goodman, et ne pas l'évoquer sans être sûr de le maîtriser. sinon donner $\alpha = \frac{-i}{\lambda}$ en justifiant que ça doit être homogène à l'inverse d'une longueur et que $\lambda$ est la longueur caract du problème.\\
Attention au terme "champ proche" qui a aujourd'hui un sens tout particulier, dire à "distance finie" et à "distance infinie".\\
Faire figure taille caractéristique de la tache par une fente A en fonction de la taille caractéristique de la fente a : Og $A=\alpha a$ et diffraction de Fraunhoffer : $A = \frac{\beta}{a}$.\\
Commenter l'analogie fréquence spatiale $\leftrightarrow$ fréquences temporelles.


\end{document}
