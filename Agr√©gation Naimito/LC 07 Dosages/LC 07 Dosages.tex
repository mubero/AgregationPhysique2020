\documentclass[12pt,prb,aps,epsf]{article}
\usepackage[utf8]{inputenc}
\usepackage{amsmath}
\usepackage{amsfonts}
\usepackage{amssymb}
\usepackage{graphicx} 
\usepackage{latexsym} 
\usepackage[toc,page]{appendix}
\usepackage{listings}
\usepackage{xcolor}
\usepackage{soul}
\usepackage[T1]{fontenc}
\usepackage{amsthm}
\usepackage{mathtools}
\usepackage{setspace}
\usepackage{array,multirow,makecell}
\usepackage{geometry}
\usepackage{textcomp}
\usepackage{float}
%\usepackage{siunitx}
\usepackage{cancel}
%\usepackage{tikz}
%\usetikzlibrary{calc, shapes, backgrounds, arrows, decorations.pathmorphing, positioning, fit, petri, tikzmark}
\usepackage{here}
\usepackage{titlesec}
%\usepackage{bm}
\usepackage{bbold}

\geometry{hmargin=2cm,vmargin=2cm}

\begin{document}
	
	\title{LC 07 Dosages}
	\author{Naïmo Davier}
	\date{Agrégation 2019}
	
	\maketitle
	
	\tableofcontents
	
	\pagebreak
	


\section{Prérequis et Introduction}
Dosage : permet de déterminer la concentration d'un soluté.\\
Il existe différents types de dosages, utilisés dans différents domaines (contrôles de qualité).

\section{Dosage par étalonnage}
\textbf{Cachau Hereillat} \textit{Des expériences de la famille Réd-Ox} p395\\

On va ici réaliser un étalonnage grâce à des solutions de concentrations connues afin de pouvoir ensuite déterminer une concentration inconnue.

\subsection{Spectrocolorimétrie : dosage d'une eau de Dakin}

On va ici doser les ions permanganate $MnO_4^-$ dans une eau de Dakin. On prend différents volumes de permanganate avec des pipettes jaugées, puis on utilise des fioles jaugées pour faire des solutions témoins de concentrations connues.

 On trace ensuite l'absorbance $\mathcal{A} = \varepsilon_0 l C$ en fonction de la concentration en permanganate des différentes solutions témoin réalisée à partir de la solution mère. A partir de cette courbe on peut, en connaissant l'absorbance de l'eau de Dakin, remonter à la concentration en permanganate de cette eau de Dakin. Il est finalement possible de comparer le résultat obtenu avec le résultat donné par le fabriquant, commenter.
 
 \subsection{Conductimétrie : dosage d'un sérum physiologique}
 \textbf{Mesplède Randon} \textit{Chimie générale et analytique}.\\
 
 Détermination de la concentration en chlorure de sodium d'un sérum physiologique. On attend 0,9\,\% de chlorure de sodium. $M_{NaCl} = 58,5$g.mol$^{-1}$.
 
 \section{Dosage d'un par titrage}
 Définition de ce qu'est un dosage par titrage, et donner les différents types : direct, indirect et en retour : regarder poly de Pierre Henry Suet.
 
 \subsubsection{Dosage direct d'un vinaigre}
 \textbf{Cachau Herreillat} \textit{Des expériences de la famille Acide Base} p259\\
 
 On dose l'acide acétique $CH_3COOH$ contenu dans un vinaigre par de l'hydroxyde de sodium $NaOH$ selon
 \begin{eqnarray}
 CH_3COOH_{(aq)} + OH^-_{(aq)} \rightarrow CH_3COO^-_{(aq)} + H_2O_{(l)} \\
 EI \hspace{1.5cm}n_1\hspace{2.1cm}0\hspace{2.5cm}0\hspace{1.7cm}exces\\
 E_{eq}\hspace{1.0cm}n_1-x\hspace{1.8cm}x\hspace{2.4cm}x\hspace{1.7cm}exces
 \end{eqnarray}
 A l'équivalence on a consommé tout l'acide acétique : $n_1=x_{eq}$.\\
 On peut utiliser différentes méthodes pour ce type d'étalonnage :
 
 la colorimétrie en utilisant un indicateur coloré : la phénolphtaléïne,
 
 La conductimétrie : avant équivalence on ajoute l'ion $Na^+$ et on forme $CH_3COO^-$, après équivalence on ajoute $Na^+$ et $HO^-$ : légère rupture de pente. 
 
 PHmétrie : on s'attend à observer une marche de PH.\\
 On a 
 \begin{eqnarray}
 C_0 = \frac{C_{b}V_{eq}}{V_0}
 \end{eqnarray}
 Notion de saut de PH, explication et illustration en direct. Méthode de la dérivée pour identifier le volume équivalent.\\
 On explicite $V_{eq}$ et on en déduit $C_0$.\\
 Applications et intérêts réels.
 
 \section*{Questions/Remarques}
 Il faut expliciter le protocole, le but de chaque manip, la verrerie utilisée...\\
 C'est la leçon type où faire un calcul d'incertitudes, notamment sur la détermination du volume équivalent.\\
 Conserver les notations.\\
 réaction quantitative = totale, rapide et unique.\\
 Pour ajouter un point en direct sur une courbe faite en préparation, le faire sur une courbe de PH dans la zone tampon pour qu'il y ait peu d'écart avec la courbe précédente.
 
 
\end{document}